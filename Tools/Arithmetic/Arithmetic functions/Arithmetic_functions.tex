\documentclass[12pt]{article}
\usepackage{import}
%===============================%
%  Packages and basic settings  %
%===============================%
\usepackage[letterpaper, left=1.25in, right=1.25in, top=1in, bottom=1in]{geometry}
\usepackage{amssymb}
\usepackage{amsmath}
\usepackage{enumitem}
\usepackage{hyperref}
\usepackage[hyperref,amsthm,amsmath,framed,thmmarks]{ntheorem}
\usepackage[capitalise,noabbrev]{cleveref}
\usepackage{tikz}
\usepackage{tikz-cd}
\usetikzlibrary{braids,arrows,decorations.markings,calc}

%====================================%
%  Theorems, environments & cleveref  %
%====================================%
\theoremstyle{plain}
\newtheorem{theorem}{Theorem}[section]
\newtheorem{proposition}[theorem]{Proposition}
\newtheorem{corollary}[theorem]{Corollary}
\newtheorem{lemma}[theorem]{Lemma}

\theoremstyle{nonumberplain}
\renewtheorem{theorem*}{Theorem}
\renewtheorem{proposition*}{Proposition}
\renewtheorem{corollary*}{Corollary}
\renewtheorem{lemma*}{Lemma}

\theoremstyle{remark}
\newtheorem{conjecture}[theorem]{Conjecture}
\newtheorem{remark}[theorem]{Remark}
\newtheorem{problem}[theorem]{Open Problem}
\newtheorem{heuristic}[theorem]{Heuristic}

\theoremstyle{nonumberplain}
\renewtheorem{conjecture*}{Conjecture}
\renewtheorem{remark*}{Remark}
\renewtheorem{problem*}{Open Problem}
\renewtheorem{heuristic*}{Heuristic}

%==================================%
%  Custom commands & environments  %
%==================================%
\newcommand{\legendre}[2]{\left(\frac{#1}{#2}\right)}
\newcommand{\dlegendre}[2]{\displaystyle{\left(\frac{#1}{#2}\right)}}
\newcommand{\tlegendre}[2]{\textstyle{\left(\frac{#1}{#2}\right)}}
\newcommand{\psum}{\sideset{}{'}\sum}
\newcommand{\asum}{\sideset{}{^{\ast}}\sum}
\newcommand{\tmod}[1]{\ (\mathrm{mod}\text{ }#1)}
\renewcommand{\bmod}[1]{\ \left(\mathrm{mod}\text{ }#1\right)}
\newcommand{\xto}[1]{\xrightarrow{#1}}
\newcommand{\xfrom}[1]{\xleftarrow{#1}}
\newcommand{\normal}{\mathrel{\unlhd}}
\newcommand{\mf}{\mathfrak}
\newcommand{\mc}{\mathcal}
\newcommand{\ms}{\mathscr}

\newcommand{\Mat}{\mathrm{Mat}}
\newcommand{\GL}{\mathrm{GL}}
\newcommand{\SL}{\mathrm{SL}}
\newcommand{\PSL}{\mathrm{PSL}}
\renewcommand{\O}{\mathrm{O}}
\newcommand{\SO}{\mathrm{SO}}
\newcommand{\U}{\mathrm{U}}
\newcommand{\Sp}{\mathrm{Sp}}

\newcommand{\N}{\mathbb{N}}
\newcommand{\Z}{\mathbb{Z}}
\newcommand{\Q}{\mathbb{Q}}
\newcommand{\R}{\mathbb{R}}
\newcommand{\C}{\mathbb{C}}
\newcommand{\F}{\mathbb{F}}
\renewcommand{\H}{\mathbb{H}}
\renewcommand{\P}{\mathbb{P}}

\renewcommand{\a}{\alpha}
\renewcommand{\b}{\beta}
\newcommand{\g}{\gamma}
\renewcommand{\d}{\delta}
\newcommand{\z}{\zeta}
\renewcommand{\t}{\theta}
\renewcommand{\i}{\iota}
\renewcommand{\k}{\kappa}
\renewcommand{\l}{\lambda}
\newcommand{\s}{\sigma}
\newcommand{\w}{\omega}

\newcommand{\G}{\Gamma}
\newcommand{\D}{\Delta}
\renewcommand{\L}{\Lambda}
\newcommand{\W}{\Omega}
\newcommand{\scL}{\mathscr{L}}

\newcommand{\e}{\varepsilon}
\newcommand{\vt}{\vartheta}
\newcommand{\vphi}{\varphi}
\newcommand{\emt}{\varnothing}

\newcommand{\x}{\times}
\newcommand{\ox}{\otimes}
\newcommand{\op}{\oplus}
\newcommand{\bigox}{\bigotimes}
\newcommand{\bigop}{\bigoplus}
\newcommand{\del}{\partial}
\newcommand{\<}{\langle}
\renewcommand{\>}{\rangle}
\newcommand{\lf}{\lfloor}
\newcommand{\rf}{\rfloor}
\newcommand{\wtilde}{\widetilde}
\newcommand{\what}{\widehat}
\newcommand{\conj}{\overline}
\newcommand{\cchi}{\conj{\chi}}

\DeclareMathOperator{\id}{\textrm{id}}
\DeclareMathOperator{\sgn}{\mathrm{sgn}}
\DeclareMathOperator{\im}{\mathrm{im}}
\DeclareMathOperator{\rk}{\mathrm{rk}}
\DeclareMathOperator{\adj}{\mathrm{adj}}
\DeclareMathOperator{\tr}{\mathrm{trace}}
\DeclareMathOperator{\nm}{\mathrm{norm}}
\DeclareMathOperator{\disc}{\mathrm{disc}}
\DeclareMathOperator{\ord}{\mathrm{ord}}
\DeclareMathOperator{\sym}{\mathrm{sym}}
\DeclareMathOperator{\ext}{\mathrm{ext}}
\DeclareMathOperator{\Hom}{\mathrm{Hom}}
\DeclareMathOperator{\End}{\mathrm{End}}
\DeclareMathOperator{\Aut}{\mathrm{Aut}}
\DeclareMathOperator{\Tor}{\mathrm{Tor}}
\DeclareMathOperator{\Ann}{\mathrm{Ann}}
\DeclareMathOperator{\Gal}{\mathrm{Gal}}
\DeclareMathOperator{\Trace}{\mathrm{Tr}}
\DeclareMathOperator{\Norm}{\mathrm{N}}
\DeclareMathOperator{\Cl}{\mathrm{Cl}}
\DeclareMathOperator{\Span}{\mathrm{Span}}
\DeclareMathOperator*{\Res}{\mathrm{Res}}
\DeclareMathOperator{\Vol}{\mathrm{Vol}}
\DeclareMathOperator{\Li}{\mathrm{Li}}
\DeclareMathOperator{\Supp}{\mathrm{Supp}}
\renewcommand{\Re}{\mathrm{Re}}
\renewcommand{\Im}{\mathrm{Im}}
\DeclareMathOperator{\Ph}{\mathrm{Ph}}
\DeclareMathOperator{\SC}{\mathrm{SC}}


\newcommand{\GH}{\G\backslash\H}
\newcommand{\GG}{\G_{\infty}\backslash\G}

\newenvironment{psmallmatrix}
  {\left(\begin{smallmatrix}}
  {\end{smallmatrix}\right)}

\newcommand{\smc}[1]{
    \mathchoice
    {{\scriptstyle\mathcal{#1}}}
    {{\scriptstyle\mathcal{#1}}}
    {{\scriptscriptstyle\mathcal{#1}}}
    {\scalebox{0.7}{$\scriptscriptstyle\mathcal{#1}$}}
}

%============%
%  Comments  %
%============%
\newcommand{\todo}[1]{\textcolor{red}{\sf Todo: [#1]}}

%===================%
%  Label reminders  %
%===================%
% [label=(\roman*)]
% [label=(\alph*)]
% [label=(\arabic{enumi})]

%==================%
%  Other settings  %
%==================%
\pgfdeclarelayer{background}
\pgfsetlayers{background,main}
\tikzset{->-/.style={decoration={
  markings,
  mark=at position .5 with {\arrow{>}}},postaction={decorate}}}

%=================%
%  Title & Index  %
%=================%
\title{Arithmetic functions and Dirichlet characters}
\author{Henry Twiss}
\makeindex

\begin{document}
  \date{}
  \maketitle
  \section{Arithmetic Functions}
    An arithmetic function \(f\) is a function \(f:\Z_{\ge 1} \to \C\). That is, it takes the positive integers into the complex numbers. We say that \(f\) is \textbf{additive}\index{additive} if \(f(nm) = f(n)+f(m)\) for all positive integers \(n\) and \(m\) such that \((n,m) = 1\). If this condition simply holds for all \(n\) and \(m\) then we say \(f\) is \textbf{completely additive}\index{completely additive}. Similarly, we say that \(f\) is \textbf{multiplicative}\index{multiplicative} if \(f(nm) = f(n)f(m)\) for all positive integers \(n\) and \(m\) such that \((n,m) = 1\). If this condition simply holds for all \(n\) and \(m\) then we say \(f\) is \textbf{completely multiplicative}\index{completely multiplicative}. Many important arithmetic functions are either additive, completely additive, multiplicative, or completely multiplicative. Note that if a \(f\) is additive or multiplicative then \(f\) is uniquely determined by its values on prime powers and if \(f\) is completely additive or completely multiplicative then it is uniquely determined by its values on primes. Moreover, if \(f\) is additive or completely additive then \(f(1) = 0\) and if \(f\) is multiplicative or completely multiplicative then \(f(1) = 1\). Below is a list defining the most important arithmetic functions (some of these functions are restrictions of common functions but we define them here as arithmetic functions because their domain being \(\Z_{\ge 1}\) is important):
    \begin{enumerate}[label*=(\roman*)]
      \item The \textbf{constant function}\index{constant function}: The function \(\mathbf{1}(n)\) restricted to all \(n \ge 1\). This function is neither additive or multiplicative.
      \item The \textbf{indicator function}\index{indicator function}: The function \(\d(n)\) defined by
      \[
        \d(n) = \begin{cases} 1 & \text{if \(n = 1\)}, \\ 0 & \text{if \(n \ge 2\)}. \end{cases}
      \]
      This function is completely multiplicative.
      \item The \textbf{identity function}\index{identity function}: The function \(\id(n)\) restricted to all \(n \ge 1\). This function is completely multiplicative.
      \item The \textbf{logarithm}\index{logarithm function}: The function \(\log(n)\) restricted to all \(n \ge 1\). This function is completely additive.
      \item The \textbf{M\"obius function}\index{M\"obius function}: The function \(\mu(n)\) defined by
      \[
        \mu(n) = \begin{cases} 1 & \text{if \(n\) is square-free with an even number of prime factors}, \\ -1 & \text{if \(n\) is square-free with an odd number of prime factors}, \\ 0 & \text{if \(n\) is not square-free}, \end{cases}
      \]
      for all \(n \ge 1\). This function is multiplicative.
      \item The \textbf{characteristic function of square-free integers}\index{characteristic function of square-free integers}: The square of the M\"obius function \(\mu^{2}(n)\) for all \(n \ge 1\). This function is multiplicative.
      \item \textbf{Liouville's function}\index{Liouville's function}: The function \(\l(n)\) defined by
      \[
        \l(n) = \begin{cases} 1 & \text{if \(n = 1\)}, \\ (-1)^{k} & \text{if \(n\) is composed of \(k\) not necessarily distinct prime factors}, \end{cases}
      \]
      for all \(n \ge 1\). This function is completely multiplicative.
      \item \textbf{Euler's totient function}\index{Euler's totient function}: The function \(\vphi(n)\) defined by
      \[
        \vphi(n) = \psum_{m \tmod{n}}1,
      \]
      for all \(n \ge 1\). This function is multiplicative.
      \item The \textbf{divisor function}\index{divisor function}: The function \(\s_{0}(n)\) defined by
      \[
        \s_{0}(n) = \sum_{d \mid n}1,
      \]
      for all \(n \ge 1\). This function is multiplicative.
      \item The \textbf{sum of divisors function}\index{sum of divisors function}: The function \(\s_{1}(n)\) defined by
      \[
        \s_{1}(n) = \sum_{d \mid n}d,
      \]
      for all \(n \ge 1\). This function is multiplicative.
      \item The \textbf{generalized sum of divisors function}\index{generalized sum of divisors function}: The function \(\s_{s}(n)\) defined by
      \[
        \s_{s}(n) = \sum_{d \mid n}d^{s},
      \]
      for all \(n \ge 1\) and any \(s \in \C\). This function is multiplicative.
      \item The \textbf{number of distinct prime factors function}\index{number of distinct prime factors function}: The function \(\w(n)\) defined by
      \[
        \w(n) = \sum_{p \mid n}1,
      \]
      for all \(n \ge 1\). This function is additive.
      \item The \textbf{total number of prime divisors function}\index{total number of prime divisors function}: The function \(\W(n)\) defined by
      \[
        \W(n) = \sum_{p ^{m} \mid n}1,
      \]
      for all \(n \ge 1\) and where \(m \ge 1\). This function is completely additive.
      \item The \textbf{von Mangoldt function}\index{von Mangoldt function}: The function \(\L(n)\) defined by
      \[
        \L(n) = \begin{cases} 0 & \text{if \(n\) is not a prime power}, \\ \log(p) & \text{if \(n = p^{m}\) for some prime \(p\) and integer \(m \ge 1\)}, \end{cases}
      \]
      for all \(n \ge 1\). This function is neither additive or multiplicative.
    \end{enumerate}
  \section{Dirichlet Convolution and the M\"obius Inversion Formula}
    If \(f\) and \(g\) are two arithmetic functions then we can define a new arithmetic function \(f \ast g\) called the \textbf{Dirichlet convolution}\index{Dirichlet convolution} of \(f\) and \(g\) defined by
    \[
      (f \ast g)(n) = \sum_{d \mid n}f(d)g\left(\frac{n}{d}\right),
    \]
    for all \(n \ge 1\). Note that \(f \ast g = g \ast f\). This is especially useful when \(f\) and \(g\) are multiplicative:

    \begin{proposition}\label{prop:Dirichlet_convolution_of_multiplicative_functions}
      If \(f\) and \(g\) are multiplicative arithmetic functions then so is their Dirichlet convolution \(f \ast g\).
    \end{proposition}
    \begin{proof}
      Let \(n,m \ge 1\) be such that \((n,m) = 1\). Then every \(d \mid nm\) is of the form \(d = d'd''\) with \(d' \mid n\), \(d'' \mid m\), and \((d',d'') = 1\). Then
      \begin{align*}
        (f \ast g)(nm) &= \sum_{d \mid nm}f(d)g\left(\frac{nm}{d}\right) \\
        &= \sum_{\substack{d' \mid n \\ d'' \mid m \\ (d',d'') = 1}}f(d'd'')g\left(\frac{nm}{d'd''}\right) \\
        &= \sum_{\substack{d' \mid n \\ d'' \mid m \\ (d',d'') = 1}}f(d')f(d'')g\left(\frac{n}{d'}\right)g\left(\frac{m}{d''}\right) \\
        &= \left(\sum_{d' \mid n}f(d')g\left(\frac{n}{d'}\right)\right)\left(\sum_{d'' \mid m}f(d'')g\left(\frac{m}{d''}\right)\right) \\
        &= (f \ast g)(n)(f \ast g)(m).
      \end{align*}
      This completes the proof.
    \end{proof}

    From \cref{prop:Dirichlet_convolution_of_multiplicative_functions} we see that Dirichlet convolution makes the set of multiplicative functions into a semigroup. In fact, it is a monoid since the indicator function \(\d\) acts as an identity. Indeed, for any multiplicative function \(f\), we have
    \[
      (f \ast \d)(n) = \sum_{d \mid n}f(d)\d\left(\frac{n}{d}\right) = f(n).
    \]
    Therefore
    \[
      f \ast \d = f.
    \]
    A certain case of interest for Dirichlet convolution is when the M\"obius function is convolved with the constant function:

    \begin{proposition}\label{prop:Mobius_indicator}
    We have
    \[
      \sum_{d \mid n}\mu(d) = \sum_{d \mid n}\mu\left(\frac{n}{d}\right) = \begin{cases} 1 & \text{if \(n = 1\)}, \\ 0 & \text{if \(n \ge 2\)}. \end{cases}
    \]
    In particular,
    \[
      \mu \ast \mathbf{1} = \mathbf{1} \ast \mu = \d.
    \]
    \end{proposition}
    \begin{proof}
      In terms of Dirichlet convolutions, the second statement is equivalent to the first (actually just \(\mu \ast \mathbf{1} = \d\) or \(\d = \mathbf{1} \ast \mu\) since Dirichlet convolution is associative). So it suffices to prove the first statement only. By \cref{prop:Dirichlet_convolution_of_multiplicative_functions}, \(\sum_{d \mid n}\mu(d)\) is multiplicative so we may assume that \(n = p^{r}\) for some prime \(p\) and \(r \ge 0\). When \(r = 0\), \(d = 1\) and so the sum is \(1\). For \(r \ge 1\), \(d\) runs over \(1,p,\ldots,p^{r}\) and the only \(d\) for which \(\mu(d) \neq 0\) is \(d = 1\) and \(d = p\). But \(\mu(1)+\mu(p) = 0\) so that the sum is zero in this case. This completes the proof. 
    \end{proof}

    With \cref{prop:Mobius_indicator}, we can prove the infamous \textbf{M\"obius inversion formula}\index{M\"obius inversion formula}:

    \begin{theorem*}[M\"obius inversion formula]
      Suppose \(f\) and \(g\) are arithmetic functions. Then
      \[
        g(n) = \sum_{d \mid n}f(d),
      \]
      for all \(n \ge 1\), if and only if
      \[
        f(n) = \sum_{d \mid n}g(d)\mu\left(\frac{n}{d}\right),
      \]
      for all \(n \ge 1\). In particular,
      \[
        g = f \ast \mathbf{1},
      \]
      if and only if
      \[
        f = g \ast \mu.
      \]
    \end{theorem*}
    \begin{proof}
      In terms of Dirichlet convolutions, the second statement is equivalent to the first. So it suffices to prove the second statement.
      Convolving the first formula with \(\mu\) gives
      \[
        g \ast \mu = f \ast \mathbf{1} \ast \mu = f \ast \d = f,
      \]
      with the last two equalities following from \cref{prop:Mobius_indicator} and that \(\d\) is the identity for Dirichlet convolution. This proves the forward implication. The reverse implication follows by convolving the second formula with \(\mathbf{1}\) and arguing analogously. 
    \end{proof}
  \section{Specially Multiplicative Functions}
    Let \(f\) be a multiplicative arithmetic function. We associate to \(f\) the completely multiplicative function \(f_{A}\) defined on primes \(p\) by
    \[
      f_{A}(p) = f(p^{2})-f(p)^{2}.
    \]
    We say that \(f\) is \textbf{specially multiplicative}\index{specially multiplicative} if
    \[
        f(n)f(m) = \sum_{d \mid (n,m)}f_{A}(d)f\left(\frac{nm}{d^{2}}\right),
    \]
    for all \(n,m \ge 1\). In fact, this is equivalent to the identity
    \[
        f(nm) = \sum_{d \mid (n,m)}\mu(d)f_{A}(d)f\left(\frac{n}{d}\right)f\left(\frac{m}{d}\right),
    \]
    for all \(n,m \ge 1\) as the following proposition shows:

    \begin{proposition}\label{prop:specially_multiplicative_functions}
      Let \(f\) be a multiplicative function. Then
      \[
        f(n)f(m) = \sum_{d \mid (n,m)}f_{A}(d)f\left(\frac{nm}{d^{2}}\right),
      \]
      for all \(n,m \ge 1\), if and only if
      \[
        f(nm) = \sum_{d \mid (n,m)}\mu(d)f_{A}(d)f\left(\frac{n}{d}\right)f\left(\frac{m}{d}\right),
      \]
      for all \(n,m \ge 1\).
    \end{proposition}
    \begin{proof}
      Suppose the first identity holds. Then by \cref{prop:Mobius_indicator}, we have
      \begin{align*}
        f(nm) &= \sum_{d \mid (n,m)}f_{A}(d)f\left(\frac{nm}{d^{2}}\right)\sum_{e \mid d}\mu\left(\frac{d}{e}\right) \\
        &= \sum_{\substack{d \mid (n,m) \\ e \mid d}}\mu\left(\frac{d}{e}\right)f_{A}(d)f\left(\frac{nm}{d^{2}}\right) \\
        &= \sum_{\substack{d \mid (n,m) \\ e \mid \left(\frac{n}{d},\frac{m}{d}\right)}}\mu(d)f_{A}(de)f\left(\frac{nm}{(de)^{2}}\right) && \text{\(d \to de\)} \\
        &= \sum_{d \mid (n,m)}\mu(d)f_{A}(d)\sum_{e \mid \left(\frac{n}{d},\frac{m}{d}\right)}f_{A}(e)f\left(\frac{\frac{n}{d}\frac{m}{d}}{e^{2}}\right) \\
        &= \sum_{d \mid (n,m)}\mu(d)f_{A}(d)f\left(\frac{n}{d}\right)f\left(\frac{m}{d}\right),
      \end{align*}
      where the last equality holds by the first identity. This proves the forward implication. For the reverse implication, suppose the second identity holds. By \cref{prop:Mobius_indicator} again, we have
      \begin{align*}
        f(n)f(m) &= \sum_{d \mid (n,m)}f_{A}(d)f\left(\frac{n}{d}\right)f\left(\frac{m}{d}\right)\sum_{e \mid d}\mu(e) \\
        &= \sum_{\substack{d \mid (n,m) \\ e \mid d}}\mu(e)f_{A}(d)f\left(\frac{n}{d}\right)f\left(\frac{m}{d}\right) \\
        &= \sum_{\substack{d \mid (n,m) \\ e \mid \left(\frac{n}{d},\frac{m}{d}\right)}}\mu(e)f_{A}(de)f\left(\frac{n}{de}\right)f\left(\frac{m}{de}\right) && \text{\(d \to de\)} \\
        &= \sum_{d \mid (n,m)}f_{A}(d)\sum_{e \mid \left(\frac{n}{d},\frac{m}{d}\right)}\mu(e)f_{A}(e)f\left(\frac{\frac{n}{d}}{e}\right)f\left(\frac{\frac{m}{d}}{e}\right) \\
        &= \sum_{d \mid (n,m)}f_{A}(d)\sum_{e \mid \left(\frac{n}{d},\frac{m}{d}\right)}f\left(\frac{nm}{d^{2}}\right),
      \end{align*}
      where the last equality holds by the second identity. This proves the reverse implication completing the proof.
    \end{proof}
  \section{Asymptotics of Arithmetic Functions}
    We will discuss the growth of \(\s_{0}(n)\). Firstly, we generalize \(\s_{0}(n)\), \(\s_{1}(n)\), and \(\s_{s}(n)\) to all nonzero \(n \in \Z\) by setting
    \[
      \s_{0}(n) = \s_{0}(|n|), \quad \s_{1}(n) = \s_{1}(|n|), \quad \text{and} \quad \s_{s}(n) = \s_{s}(|n|),
    \]
    for all \(s \in \C\). It is very useful to know that \(\s_{0}(n)\) grows slowly:

    \begin{proposition}\label{prop:sum_of_divisors_growth_rate}
      \phantom{ }
      \[
        \s_{0}(n) \ll_{\e} n^{\e}.
      \]
    \end{proposition}
    \begin{proof}
      Let \(n = p_{1}^{r_{1}} \cdots p_{k}^{r_{k}}\) be the prime factorization of \(n\). As \(\s_{0}(p^{r_{i}}) = r_{i}+1\) for \(1 \le i \le k\), multiplicativity implies \(\s_{0}(n) = (r_{1}+1) \cdots (r_{k}+1)\). Then
      \[
        \frac{\s_{0}(n)}{n^{\e}} = \prod_{1 \le i \le r}\frac{r_{i}+1}{p_{i}^{r_{i}\e}}.
      \]
      It suffices to show that the right-hand side is bounded by a positive constant \(c(\e)\). For a prime \(p\), consider the nonnegative continuous function \(f_{p,\e}(x)\) defined by
      \[
        f_{p,\e}(x) = \frac{x+1}{p^{x\e}},
      \]
      for \(x \ge -1\). The derivative is given by
      \[
        f_{p,\e}'(x) = \frac{1-\log(p^{\e})(x+1)}{p^{x\e}},
      \]
      which is negative for \(1 < \log(p^{\e})(x+1)\) or equivalently \(\frac{1}{\log(p^{\e})}-1 < x\). Therefore \(f_{p,\e}(x)\) is eventually decreasing and so attains a maximum positive value. In particular, it attains a maximal positive integral value for some \(a_{p} \ge 0\) because \(f_{p,\e}(-1) = 0\) and \(f_{p,\e}(x)\) is positive for \(x > -1\). Then the inequalities \(f_{p,\e}(a_{p}-1) \le f_{p,\e}(a_{p})\) and \(f_{p,\e}(a_{p}+1) \le f_{p,\e}(a_{p})\) are
      \[
        \frac{a_{p}}{p^{(a_{p}-1)\e}} \le \frac{a_{p}+1}{p^{a_{p}\e}} \quad \text{and} \quad \frac{a_{p}+2}{p^{(a_{p}+1)\e}} \le \frac{a_{p}+1}{p^{a_{p}\e}},
      \]
      respectively. Upon isolating \(a_{p}\) to obtain an upper bound in the first inequality and a lower bound in the second inequality, we find that
      \[
        \frac{1}{p^{\e}-1}-1 \le a_{p} \le \frac{1}{p^{\e}-1}.
      \]
      Therefore we may take \(a_{p} = \left\lfloor \frac{1}{p^{\e}-1}-1 \right\rfloor\). Moreover,
      \[
        \frac{\frac{1}{p^{\e}-1}}{p^{\e\left(\frac{1}{p^{\e}-1}-1\right)}} \le f_{p}(a_{p}) \le \frac{\frac{1}{p^{\e}-1}+1}{p^{\e\left(\frac{1}{p^{\e}-1}\right)}},
      \]
      and taking the logarithm yields
      \[
        \log\left(\frac{1}{p^{\e}-1}\right)-\log(p^{\e})\left(\frac{1}{p^{\e}-1}-1\right) \le \log(f_{p}(a_{p})) \le \log\left(\frac{1}{p^{\e}-1}+1\right)-\log(p^{\e})\left(\frac{1}{p^{\e}-1}\right),
      \]
      which can be further expressed as
      \[
        \log(p^{\e})-\log(p^{\e}-1)-\log(p^{\e})\left(\frac{1}{p^{\e}-1}\right) \le \log(f_{p}(a_{p})) \le \log(p^{\e})-\log(p^{\e}-1)-\log(p^{\e})\left(\frac{1}{p^{\e}-1}\right),
      \]
      Taking the limit as \(p \to \infty\) and using L'H\^opital's rule for the \(\log(p^{\e})\left(\frac{1}{p^{\e}-1}\right)\) terms, the left-hand and right-hand sides both tend to zero. Therefore \(\log(f_{p}(a_{p})) \to 0\) and hence \(f_{p}(a_{p}) \to 1\). This guarantees that the infinite product \(\prod_{p}f_{p,\e}(a_{p})\) converges to some positive value \(c(\e)\). But then
      \[
        \prod_{1 \le i \le r}\frac{r_{i}+1}{p_{i}^{r_{i}\e}} \le c(\e),
      \]
      from which the claim follows.
    \end{proof}
  \section{Dirichlet Characters}
    Let \(m\) be a positive integer. A multiplicative homomorphism
    \[
      \chi:\Z \to \C,
    \]
    is said to be a \textit{Dirichlet character} modulo \(m\) if it is \(m\)-periodic and such that \(\chi(a) = 0\) if and only if \((a,m) > 1\). We call \(m\) the \textit{modulus} of \(\chi\). A Dirichlet character is necessarily a completely multiplicative arithmetic function when restricted to the positive integers.
    
    We say a Dirichlet character \(\chi\) is \textit{principal} if it only takes values \(0\) or \(1\). There is always a unique principal Dirichlet character modulo \(m\), denoted \(\chi_{m,0}\), defined by
    \[
      \chi_{m,0}(a) = \begin{cases} 1 & (a,m) = 1, \\ 0 & (a,m) > 1. \end{cases}
    \]
    When the modulus is \(1\), the principal Dirichlet character is identically \(1\) and we call this the \textit{trivial Dirichlet character}. This is the only Dirichlet character modulo \(1\).
    
    By Euler's little theorem, \(a^{\vphi(m)} \equiv 1 \tmod{m}\) provided \((a,m) = 1\). This forces \(\chi(a)^{\vphi(m)} = 1\) and so the nonzero values of any Dirichlet character modulo \(m\) are \(\vphi(m)\)-th roots of unity. This implies that there are only finitely many Dirichlet characters of any fixed modulus. Given two Dirichlet character \(\chi\) and \(\psi\) modulo \(m\), the functions
    \[
      \chi\psi:\Z \to \C \quad \text{and} \quad \cchi:\Z \to \C,
    \]
    are also Dirichlet characters modulo \(m\). This turns the set of such Dirichlet characters into an abelian group denote by \(X_{m}\) where the identity is the principal Dirichlet character modulo \(m\). We call \(\cchi\) the \textit{conjugate} of \(\chi\). Since the nonzero values of \(\chi\) are roots of unity, the conjugate is necessarily the inverse. 
    
    This is all strikingly similar to characters on \((\Z/m\Z)^{\ast}\) and there is indeed a connection. By the multiplicativity and \(m\)-periodicity of \(\chi\), it induces a character of \((\Z/m\Z)^{\ast}\). Conversely, if we are given a character on \((\Z/m\Z)^{\ast}\) we can extend it to a Dirichlet character \(\chi\) by defining it to be \(m\)-periodic and declaring \(\chi(a) = 0\) if \((a,m) > 1\). We call this extension a \textit{zero extension}. This argument shows that Dirichlet characters modulo \(m\) are exactly the zero extensions of characters on \((\Z/m\Z)^{\ast}\). As abelian groups are isomorphic to their character groups, we deduce that the group of Dirichlet characters modulo \(m\) is isomorphic to \((\Z/m\Z)^{\ast}\). Therefore there are \(\vphi(m)\) Dirichlet characters modulo \(m\) and we identify them with the characters on \((\Z/m\Z)^{\ast}\).
    
    We now state two very useful relations called \textit{Dirichlet orthogonality relations} for Dirichlet characters: \todo{xxx}

    \begin{proposition*}[Dirichlet orthogonality relations]
    \phantom{ }
      \begin{enumerate}[label*=(\roman*)]
        \item For any two Dirichlet characters \(\chi\) and \(\psi\) modulo \(m\),
        \[
          \frac{1}{\vphi(m)}\psum_{a \tmod{m}}\chi(a)\conj{\psi}(a) = \d_{\chi,\psi}.
        \]
        \item For any \(a,b \in (\Z/m\Z)^{\ast}\),
        \[
          \frac{1}{\vphi(m)}\sum_{\chi \tmod{m}}\chi(a)\cchi(b) = \d_{a,b}.
        \]
      \end{enumerate}
    \end{proposition*}
    \begin{proof}
      We will prove the statements separately.
      \begin{enumerate}[label*=(\roman*)]
        \item Denote the left-hand side by \(S\) and let \(b\) be such that \((b,m) = 1\). Then
        \begin{align*}
          \chi(b)\conj{\psi}(b)S &= \frac{\chi(b)\conj{\psi}(b)}{\vphi(m)}\psum_{a \tmod{m}}\chi(a)\conj{\psi}(a) \\
          &= \frac{1}{\vphi(m)}\psum_{a \tmod{m}}\chi(ab)\conj{\psi}(ab) \\
          &= \frac{1}{\vphi(m)}\psum_{a \tmod{m}}\chi(a)\conj{\psi}(a) && \text{\(a \mapsto a\conj{b}\)} \\
          &= S.
        \end{align*}
        Consequently \(S = 0\) unless \(\chi(b)\conj{\psi}(b) = 1\) for all \(b\) such that \((b,m) = 1\). This happens if and only if \(\psi = \chi\) in which case \(S = 1\) proving (i).
        \item Denote the left-hand side by \(S\). Let \(\psi\) be any Dirichlet character modulo \(m\). Then
        \begin{align*}
          \psi(a)\conj{\psi}(b)S &= \frac{\psi(a)\conj{\psi}(b)}{\vphi(m)}\sum_{\chi \tmod{m}}\chi(a)\cchi(b) \\
          &= \frac{1}{\vphi(m)}\sum_{\chi \tmod{m}}\chi\psi(a)\conj{\chi\psi}(b) \\
          &= \frac{1}{\vphi(m)}\sum_{\chi \tmod{m}}\chi(a)\cchi(b) && \text{\(\chi \mapsto \chi\conj{\psi}\)} \\
          &= S.
        \end{align*}
        Thus \(S = 0\) unless \(\psi(a)\conj{\psi}(b) = \psi(a\conj{b}) = 1\) for all Dirichlet characters \(\psi\) modulo \(m\). If this happens then \(a\conj{b} = 1 \tmod{m}\). To see this, let \(m = p_{1}^{r_{1}} \cdots p_{k}^{r_{k}}\) be the prime factorization of \(m\). By the structure theorem for finite abelian groups, we have
        \[
          (\Z/m\Z)^{\ast} \cong (\Z/p_{1}^{r_{1}}\Z)^{\ast} \x \cdots \x (\Z/p_{k}^{r_{k}}\Z)^{\ast}.
        \]
        Under this isomorphism, any \(n\) taken modulo \(m\) with \((n,m) = 1\) may be written uniquely as \(n = n_{1} \cdots n_{k}\) where \(n_{i}\) is taken modulo \(p_{i}^{r_{i}}\) with \((n_{i},p_{i}^{r_{i}}) = 1\) for \(1 \le i \le k\). Let \(\w_{i}\) be a primitive \(p_{i}^{r_{i}}\)-th root of unity for all \(i\) and set
        \[
          \psi(n) = \w_{1}^{n_{1}} \cdots \w_{k}^{n_{k}}.
        \]
        Clearly \(\psi\) is a character of \((\Z/m\Z)^{\ast}\) and is therefore a Dirichlet character modulo \(m\). Writing \(a = e_{1} \cdots e_{k}\) and \(b = f_{1} \cdots f_{k}\) under this isomorphism, it follows that
        \[
          1 = \w_{1}^{e_{1}\conj{f_{1}}} \cdots \w_{k}^{e_{k}\conj{f_{k}}},
        \]
        since \(\psi(a\conj{b}) = 1\). As \(\w_{i}\) has order \(p_{i}^{r_{i}}\) and \(1 \le e_{i},f_{i} \le p_{i}^{r_{i}}-1\) for all \(i\), the only way the above identity holds is if \(e_{i} \equiv f_{i} \tmod{p_{i}^{r_{i}}}\) for all \(i\). This implies \(a\conj{b} = 1 \tmod{m}\). But then \(S = 1\) and (ii) follows.
      \end{enumerate}
    \end{proof}

    In many practical settings, the Dirichlet orthogonality relations are often used in the following form:

    \begin{corollary}\label{cor:Dirichlet_orthogonality_relations}
    \phantom{ }
      \begin{enumerate}[label*=(\roman*)]
        \item For any Dirichlet character \(\chi\) modulo \(m\),
        \[
          \frac{1}{\vphi(m)}\psum_{a \tmod{m}}\chi(a) = \d_{\chi,\chi_{m,0}}.
        \]
        \item For any \(a \in (\Z/m\Z)^{\ast}\),
        \[
          \frac{1}{\vphi(m)}\sum_{\chi \tmod{m}}\chi(a) = \d_{a,1}.
        \]
      \end{enumerate}
    \end{corollary}
    \begin{proof}
      For (i), take \(\psi = \chi_{m,0}\) in the Dirichlet orthogonality relations (namely (i)). For (ii), take \(b \equiv 1 \tmod{m}\) in the Dirichlet orthogonality relations (namely (ii)).
    \end{proof}

    We will now describe how Dirichlet characters of a fixed modulus arise from Dirichlet characters of a smaller modulus. Let \(\chi_{m_{1}}\) and \(\chi_{m_{2}}\) be Dirichlet characters modulo \(m_{1}\) and \(m_{2}\). If \(m_{1} \mid m_{2}\) then \((a,m_{2}) = 1\) implies \((a,m_{1}) = 1\). Accordingly, we say \(\chi_{m_{2}}\) is \textit{induced} from \(\chi_{m_{1}}\) (or that \(\chi_{m_{1}}\) \textbf{lifts} to \(\chi_{m_{2}}\)) if
    \[
      \chi_{m_{2}}(a) = \begin{cases} \chi_{m_{1}}(a) & \text{if \((a,m_{2}) = 1\)}, \\ 0 & \text{if \((a,m_{2}) > 1\)}. \end{cases}
    \]
    All this means is that \(\chi_{m_{2}}\) is a Dirichlet character modulo \(m_{2}\) whose values are given by those of \(\chi_{m_{1}}\). Clearly every Dirichlet character is induced from itself. On the other hand, if there is a prime \(p\) dividing \(m_{2}\) and not \(m_{1}\) (so \(m_{2}\) is a larger modulus), \(\chi_{m_{2}}\) will be different from \(\chi_{m_{1}}\) since \(\chi_{m_{2}}(p) = 0\) but \(\chi_{m_{1}}(p) \neq 0\). In general, we say a Dirichlet character is \textit{primitive} if it is not induced by any character other than itself and \textit{imprimitive} otherwise. Notice that the principal Dirichlet characters are precisely those Dirichlet characters induced from the trivial Dirichlet character, and the only primitive one is the trivial Dirichlet character. It is not a hard matter to determine when Dirichlet characters are induced:

    \begin{proposition}\label{prop:Dirichlet_character_induction_classification}
      A Dirichlet character \(\chi_{m_{2}}\) is induced from a Dirichlet character \(\chi_{m_{1}}\) if and only if \(\chi_{m_{2}}\) is constant on the residue classes in \((\Z/m_{2}\Z)^{\ast}\) that are congruent modulo \(m_{1}\). When this happens, \(\chi_{m_{1}}\) is uniquely determined.
    \end{proposition}
    \begin{proof}
      For the forward implication, if \(\chi_{m_{2}}\) is induced from \(\chi_{m_{1}}\) then \(\chi_{m_{2}}\) is constant on the residue classes in \((\Z/m_{2}\Z)^{\ast}\) that are congruent modulo \(m_{1}\) because \(\chi_{m_{1}}\) is. For the reverse implication, first note that the surjective homomorphism \(\Z/m_{2}\Z \to \Z/m_{1}\Z\) given by reduction modulo \(m_{1}\) induces a surjective homomorphism \((\Z/m_{2}\Z)^{\ast} \to (\Z/m_{1}\Z)^{\ast}\) (because reduction modulo \(m_{1}\) preserve inverses). Now suppose \(\chi_{m_{2}}\) is constant on the residue classes in \((\Z/m_{2}\Z)^{\ast}\) that are congruent modulo \(m_{1}\). Surjectivity of the previously mentioned map implies that \(\chi_{m_{2}}\) induces a unique character on \((\Z/m_{1}\Z)^{\ast}\) and hence a unique Dirichlet character modulo \(m_{1}\). By construction \(\chi_{m_{2}}\) is induced from \(\chi_{m_{1}}\).
    \end{proof}

    We are interested in primitive Dirichlet characters because they are the building blocks for all Dirichlet characters:

    \begin{theorem}\label{thm:Dirichlet_character_conductor_existance}
      Every Dirichlet character \(\chi\) is induced from a primitive Dirichlet character \(\wtilde{\chi}\) that is uniquely determined by \(\chi\).
    \end{theorem}
    \begin{proof}
      Let the modulus of \(\chi\) be \(m\). Define a partial ordering on the set of Dirichlet characters where \(\psi \le \chi\) if \(\chi\) is induced from \(\psi\). This ordering is clearly reflexive, and it is transitive by \cref{prop:Dirichlet_character_induction_classification}. Set
      \[
        X = \left\{\psi \in \bigcup_{d \mid m}X_{d}:\psi \le \chi\right\}
      \]
      This set is nonempty and finite by \cref{prop:Dirichlet_character_induction_classification}. Now suppose \(\chi_{m_{1}},\chi_{m_{2}} \in X\). Set \(m_{3} = (m_{1},m_{2})\). Also from \cref{prop:Dirichlet_character_induction_classification}, \(\chi\) is constant on the residue classes of \((\Z/m\Z)^{\ast}\) that are congruent modulo \(m_{1}\) or \(m_{2}\) and hence also \(m_{3}\). Therefore \cref{prop:Dirichlet_character_induction_classification} implies there is a unique Dirichlet character \(\chi_{m_{3}}\) modulo \(m_{3}\) that lifts to \(\chi_{m_{1}}\) and \(\chi_{m_{2}}\). We have now shown that every pair \(\chi_{m_{1}},\chi_{m_{2}} \in X\) has a lower bound \(\chi_{m_{3}}\). Hence \(X\) contains a primitive Dirichlet character \(\wtilde{\chi}\) that is minimal with respect to this partial ordering. There is only one such element. Indeed, since \(m_{3} \le m_{1},m_{2}\) the partial ordering is compatible with the total ordering by modulus. Thus \(\wtilde{\chi}\) is unique.
    \end{proof}

    In light of \cref{thm:Dirichlet_character_conductor_existance}, we define \textit{conductor} \(q\) of a Dirichlet character \(\chi\) modulo \(m\) to be the modulus of the unique primitive character \(\wtilde{\chi}\) that induces \(\chi\). This is the most important data of a Dirichlet character since it tells us how \(\chi\) is built. Note that \(\chi\) is primitive if and only if its conductor and modulus are equal. Also observe that if \(\chi\) has conductor \(q\) then \(\chi\) is actually \(q\)-periodic, we must have \(q \mid m\), and the nonzero values of \(\chi\) are all \(q\)-th roots of unity because those are the nonzero values of \(\wtilde{\chi}\). Note that \(\chi = \wtilde{\chi}\chi_{\frac{m}{q},0}\) by the definition of induced Dirichlet characters. Moreover, we have the formula
    \[
      \phi(m) = \sum_{d \mid m}N(d),
    \]
    where \(N(d)\) is the number of primitive Dirichlet characters modulo \(d\). Indeed, the right-hand side counts the number of Dirichlet characters modulo \(m\) since every such Dirichlet character is induced from a unique primitive Dirichlet character by \cref{thm:Dirichlet_character_conductor_existance} whose modulus must divide \(m\) as we have already mentioned. The right-hand side also counts the number of Dirichlet characters modulo \(m\) since we have already seen that there are \(\phi(m)\) of them (because the group of Dirichlet characters is isomorphic to the character group of \((\Z/m\Z)^{\ast}\)). Primitive Dirichlet characters also behave well with respect to multiplication if the conductors are relatively prime as the following proposition shows:

    \begin{proposition}\label{prop:primitive_characters_multiplicative_relatively_prime}
      Suppose \(\chi_{1}\) and \(\chi_{2}\) are Dirichlet characters modulo \(q_{1}\) and \(q_{2}\) respectively with \(q_{1}\) and \(q_{2}\) relatively prime. Set \(\chi = \chi_{1}\chi_{2}\) so that \(\chi\) is a Dirichlet character modulo \(q_{1}q_{2}\). Then \(\chi\) is a primitive if and only if \(\chi_{1}\) and \(\chi_{2}\) are both primitive.
    \end{proposition}
    \begin{proof}
      First suppose \(\chi\) is primitive of conductor \(q\). If \(d_{1}\) and \(d_{2}\) are the conductors of \(\chi_{1}\) and \(\chi_{2}\) respectively then \(\chi\) is \(d_{1}d_{2}\)-periodic and primitivity further implies that \(q \mid d_{1}d_{2}\). But as \(d_{1} \mid q_{1}\), \(d_{2} \mid q_{2}\), and \(q = q_{1}q_{2}\), we must have \(q = d_{1}d_{2}\) and hence \(d_{1} = q_{1}\) and \(d_{2} = q_{2}\). It follows that \(\chi_{1}\) and \(\chi_{2}\) are both primitive. Conversely, suppose \(\chi_{1}\) and \(\chi_{2}\) are both primitive. If \(d\) is the conductor of \(\chi\), set \(d_{1} = (d,q_{1})\) and \(d_{2} = (d,q_{2})\). As \((q_{1},q_{2}) = 1\) and \(q = q_{1}q_{2}\), we must have \((d_{1},d_{2}) = 1\) and \(d_{1}d_{2} = q\). But then \(d_{1} = q_{1}\) and \(d_{2} = q_{2}\). Hence \(d = q_{1}q_{2}\) which implies that \(\chi\) is primitive.
    \end{proof}
    
    We would now like to distinguish Dirichlet characters whose nonzero values are either real or imaginary. We say \(\chi\) is \textit{real} if it is real-valued. Hence the nonzero values of \(\chi\) are \(1\) or \(-1\) since they must be roots of unity. We say \(\chi\) is an \textit{complex} if it is not real. More commonly, we distinguish Dirichlet characters modulo \(m\) by their order as an element of \((\Z/m\Z)^{\ast}\). If \(\chi\) is of order \(2\), \(3\), etc.\ in \((\Z/m\Z)^{\ast}\) then we say it is \textit{quadratic}, \textit{cubic}, etc. In particular, a Dirichlet character is quadratic if and only if it is real. For any Dirichlet character \(\chi\), \(\chi(-1) = \pm 1\) because \(\chi(-1)^{2} = 1\). We would like to distinguish this parity. Accordingly, we say \(\chi\) is \textit{even} if \(\chi(-1) = 1\) and \textit{odd} if \(\chi(-1) = -1\). Clearly even Dirichlet characters are even functions and odd Dirichlet characters are odd functions. Moreover, \(\chi\) and \(\cchi\) have the same parity and any lift of \(\chi\) has the same parity as \(\chi\). Also note that
    \[
      \frac{\chi(1)-\chi(-1)}{2} = \begin{cases} 0 & \text{if \(\chi\) is even}, \\ 1 & \text{if \(\chi\) is odd}. \end{cases}
    \]
    Lastly, we would like to discuss quadratic Dirichlet characters. We can construct quadratic Dirichlet characters using Jacobi symbols. If \(m \ge 1\) is odd, consider
    \[
      \chi_{m}(n) = \tlegendre{n}{m}.
    \]
    Clearly \(\chi_{m}\) a quadratic Dirichlet character modulo \(m\) because the Jacobi symbol is multiplicative, nonzero if and only if \((n,m) = 1\), and determined modulo \(m\). However, quadratic Dirichlet characters given by Jacobi symbols do not exhaust all possible quadratic Dirichlet characters. For this, we need to use Kronecker symbols. We say that \(D \in \Z\) is a \textit{fundamental discriminant} if \(D\) is of the form
    \[
      D = \begin{cases} d & \text{if \(D \equiv 1 \tmod{4}\)}, \\ 4d & \text{if \(\frac{D}{4} \equiv 2,3 \tmod{4}\)}, \end{cases}
    \]
    for some square-free \(d \in \Z\). Necessarily \(d \equiv 1 \tmod{4}\) or \(d \equiv 2,3 \tmod{4}\) respectively and thus nonzero. We define the \textbf{quadratic Dirichlet character} \(\chi_{D}\) associated to the fundamental discriminant \(D\) by
    \[
      \chi_{D}(m) = \legendre{D}{m}.
    \]
    It turns out that \(\chi_{D}\) defines a primitive quadratic Dirichlet character, and exhausts all primitive quadratic Dirichlet characters, as the following theorem shows:

    \begin{theorem}\label{thm:fundamental_discriminant_character_primitive}
      If \(D\) is a fundamental discriminant and \(D \neq 1\) then \(\chi_{D}\) is a primitive quadratic Dirichlet character of conductor \(|D|\). Moreover, all primitive quadratic Dirichlet characters are of this form.
    \end{theorem}
    \begin{proof}
      We first show that \(\chi_{D}\) is a quadratic Dirichlet character of conductor \(|D|\). If \(D \equiv 1 \tmod{4}\), the sign in quadratic reciprocity is always \(1\) so that
      \[
        \chi_{D}(m) = \legendre{m}{|D|},
      \]
      and hence is a quadratic Dirichlet character modulo \(|D|\) because it is given by the Jacobi symbol. If \(\frac{D}{4} \equiv 3 \tmod{4}\), the sign in quadratic reciprocity is \(\tlegendre{-1}{m}\) which is the primitive quadratic Dirichlet character modulo \(4\) (there are only two Dirichlet characters modulo \(4\) since \(\vphi(4) = 2\) and clearly \(\legendre{-1}{m}\) is not principal) so that
      \[
        \chi_{D}(m) = \legendre{-1}{m}\legendre{m}{\left|\frac{D}{4}\right|},
      \]
      and hence is a Dirichlet character modulo \(|D|\). If \(\frac{D}{4} \equiv 2 \tmod{16}\), first observe that \(\tlegendre{D}{m} = \tlegendre{8}{m}\tlegendre{\frac{D}{8}}{m}\) where \(\tlegendre{8}{m}\) is one of the two primitive quadratic Dirichlet character modulo \(8\) (the other is \(\tlegendre{-8}{m}\) as there are four Dirichlet character modulo \(8\) because \(\vphi(8) = 4\) and the other two are the principal Dirichlet character and the Dirichlet character induced from \(\legendre{-1}{m}\) as mentioned previously). As \(\frac{D}{8} \equiv 1,3 \tmod{4}\), the sign in quadratic reciprocity is either \(1\) or \(\tlegendre{-1}{m}\) according to these two cases. Thus
      \[
        \chi_{D}(m) = \legendre{8}{m}\legendre{m}{\left|\frac{D}{8}\right|} \quad \text{or} \quad \chi_{D}(m) = \legendre{-8}{m}\legendre{m}{\left|\frac{D}{8}\right|},
      \]
      according to if \(\frac{D}{8} \equiv 1,3 \tmod{4}\) respectively, and hence is a quadratic Dirichlet character modulo \(|D|\). We can compactly express all of these cases as follows:
      \[
        \chi_{D}(m) = \begin{cases} \legendre{m}{|D|} & \text{if \(D \equiv 1 \tmod{4}\)}, \\ \legendre{-1}{m}\legendre{m}{\left|\frac{D}{4}\right|} & \text{if \(\frac{D}{4} \equiv 3 \tmod{4}\)}, \\ \legendre{8}{m}\legendre{m}{\left|\frac{D}{8}\right|} & \text{if \(\frac{D}{8} \equiv 1 \tmod{4}\)}, \\ \legendre{-8}{m}\legendre{m}{\left|\frac{D}{8}\right|} & \text{if \(\frac{D}{8} \equiv 3 \tmod{4}\)}. \end{cases}
      \]
      This shows that \(\chi_{D}\) is a quadratic Dirichlet characters modulo \(|D|\). It easily follows from the above that \(\chi_{D}\) is primitive. Indeed, we have already mentioned that the characters \(\tlegendre{-1}{m}\), \(\tlegendre{8}{m}\), and \(\tlegendre{-8}{m}\) are all primitive. Therefore, since \(D\), \(\frac{D}{4}\), and \(\frac{D}{8}\) are square-free according to their equivalences modulo \(4\) as given above, and \(D \neq 1\), it suffices to show by \cref{prop:primitive_characters_multiplicative_relatively_prime} that \(\chi_{p}\) is primitive for all primes \(p\) with \(p \neq 2\). This is immediate since \(p\) is prime and clearly \(\chi_{p}\) is not principal. We now show that every primitive quadratic Dirichlet character is of the form \(\chi_{D}\) for some fundamental discriminant \(D\). By \cref{prop:primitive_characters_multiplicative_relatively_prime}, it suffices to consider primitive quadratic Dirichlet character modulo \(q = p^{m}\) for some prime \(p\) and \(m \ge 1\). First suppose that \(p \neq 2\). Then \((\Z/q\Z)^{\ast}\) is cyclic and so every \(n \in (\Z/p^{m}\Z)^{\ast}\) is of the form \(n = v^{\nu}\) for some \(\nu \in (\Z/\vphi(p^{m})\Z)\) and where \(v\) is a generator of \((\Z/p^{m}\Z)^{\ast}\). It follows that every Dirichlet character \(\chi\) modulo \(p^{m}\) is of the form
      \[
        \chi(n) = e^{\frac{2\pi ik\nu}{\vphi(p^{m})}},
      \]
      where \(0 \le k \le \vphi(q)-1\). Indeed, this is a unique Dirichlet character for every such \(k\) and there are \(\vphi(p^{m})\) Dirichlet characters modulo \(p^{m}\) which is the same number of choices for \(k\). Moreover, \(\chi\) is primitive if and only if \(p \nmid k\) for otherwise \(\chi\) is a Dirichlet character modulo \(p^{m-1}\). Similarly, \(\chi\) is quadratic if and only if \(\frac{k}{\vphi(p^{m})}\) has at most \(2\) in its denominator which is equivalent to \(k \equiv \frac{\vphi(p^{m})}{2} \tmod{\vphi(p^{m})}\) and hence such a \(k\) exists and is unique because \(p \neq 2\). We also see that if \(\chi\) is quadratic, it is imprimitive unless \(m = 1\) for then \(\vphi(p) = p-1\) is not a multiple of \(p\). All of this is to say that there is a unique quadratic Dirichlet character modulo \(q\) and it is primitive if and only if \(q = p\). Necessarily, this unique primitive quadratic Dirichlet character modulo \(p\) is given by \(\chi_{D}\) for the fundamental discriminant \(D = p\) if \(p \equiv 1 \tmod{4}\) and \(D = -p\) if \(p \equiv 3 \tmod{4}\). Now suppose \(p = 2\) so that \(q = 2^{m}\) for some \(m \ge 1\). If \(m = 1\), \(\vphi(2) = 1\) and there are no primitive quadratic Dirichlet characters as the only Dirichlet character is principal. If \(m = 2\), \(\vphi(4) = 2\) so that there are two Dirichlet characters. They are both quadratic but only one is primitive, namely the principal Dirichlet character as well as the aforementioned primitive quadratic Dirichlet character \(\tlegendre{-1}{m}\). For \(m \ge 3\), \((\Z/2^{m}\Z)^{\ast} \cong C_{2} \x C_{2^{m-2}}\) where \(C_{2}\) and \(C_{2^{m-2}}\) are the cyclic groups of order \(2\) and \(2^{m-2}\) respectively. Therefore every \(n \in (\Z/2^{m}\Z)^{\ast}\) is of the form \(n = (-1)^{\mu}5^{\nu}\) for \(\mu \in \Z/2\Z\) and \(\nu \in \Z/2^{m-2}\Z\) (because the orders of \(-1\) and \(5\) modulo \(2^{m}\) are \(2\) and \(2^{m-2}\) respectively, with the latter case following by induction for \(m \ge 3\), and that \(\<-1\> \cap \<5\> = \{1\}\)). Then every Dirichlet character \(\chi\) modulo \(2^{m}\), for \(m \ge 3\), is of the form
      \[
        \chi(n) = e^{\frac{2\pi ij\mu}{2}}e^{\frac{2\pi ik\nu}{2^{m-2}}},
      \]
      where \(0 \le j \le 1\) and \(0 \le k \le 2^{m-2}-1\). Indeed, this is a unique Dirichlet character for every such choice of \(j\) and \(k\) and there are \(2^{m-1}\) Dirichlet characters modulo \(2^{m}\) which is the same number of choices for \(j\) and \(k\). Similarly to the case for \(p \neq 2\), \(\chi\) is primitive if and only if \(2^{m-2} \nmid k\), or equivalently, \(k\) is odd. Moreover, \(\chi\) is quadratic if and only if \(\frac{k}{2^{m-2}}\) has at most \(2\) in its denominator which is to say that \(2^{m-3} \mid k\). Therefore for a primitive quadratic Dirichlet to exist we must have \(k\) odd and \(2^{m-3} \mid k\) which can happen if and only if \(m = 3\). Then \(\phi(8) = 4\), so that there are four Dirichlet characters. They are all quadratic but only two are primitive, namely the principal Dirichlet character, the Dirichlet character induced from \(\tlegendre{-1}{m}\), and the two aforementioned primitive quadratic Dirichlet characters given by \(\tlegendre{8}{m}\) and \(\tlegendre{-8}{m}\). These three primitive quadratic Dirichlet characters are given by \(\chi_{D}\) for the fundamental discriminants \(D = -4\), \(D = 8\), and \(D = -8\) respectively. We have now shown that all primitive quadratic Dirichlet characters of prime power modulus are given by \(\chi_{D}\) for some fundamental discriminant \(D\) and thus the same follows for all primitive quadratic Dirichlet characters by \cref{prop:primitive_characters_multiplicative_relatively_prime}. This completes the proof.
    \end{proof}

    It follows from \cref{thm:fundamental_discriminant_character_primitive} that all quadratic Dirichlet characters are induced from some \(\chi_{D}\) (including \(D = 1\) since this corresponds to the trivial Dirichlet character). In particular, so too are the quadratic Dirichlet characters given by Jacobi symbols.
\end{document}