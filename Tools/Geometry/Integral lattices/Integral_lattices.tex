\documentclass[12pt]{article}
\usepackage{import}
%===============================%
%  Packages and basic settings  %
%===============================%
\usepackage[letterpaper, left=1.25in, right=1.25in, top=1in, bottom=1in]{geometry}
\usepackage{amssymb}
\usepackage{amsmath}
\usepackage{enumitem}
\usepackage{hyperref}
\usepackage[hyperref,amsthm,amsmath,framed,thmmarks]{ntheorem}
\usepackage[capitalise,noabbrev]{cleveref}
\usepackage{tikz}
\usepackage{tikz-cd}
\usetikzlibrary{braids,arrows,decorations.markings,calc}

%====================================%
%  Theorems, environments & cleveref  %
%====================================%
\theoremstyle{plain}
\newtheorem{theorem}{Theorem}[section]
\newtheorem{proposition}[theorem]{Proposition}
\newtheorem{corollary}[theorem]{Corollary}
\newtheorem{lemma}[theorem]{Lemma}

\theoremstyle{nonumberplain}
\renewtheorem{theorem*}{Theorem}
\renewtheorem{proposition*}{Proposition}
\renewtheorem{corollary*}{Corollary}
\renewtheorem{lemma*}{Lemma}

\theoremstyle{remark}
\newtheorem{conjecture}[theorem]{Conjecture}
\newtheorem{remark}[theorem]{Remark}
\newtheorem{problem}[theorem]{Open Problem}
\newtheorem{heuristic}[theorem]{Heuristic}

\theoremstyle{nonumberplain}
\renewtheorem{conjecture*}{Conjecture}
\renewtheorem{remark*}{Remark}
\renewtheorem{problem*}{Open Problem}
\renewtheorem{heuristic*}{Heuristic}

%==================================%
%  Custom commands & environments  %
%==================================%
\newcommand{\legendre}[2]{\left(\frac{#1}{#2}\right)}
\newcommand{\dlegendre}[2]{\displaystyle{\left(\frac{#1}{#2}\right)}}
\newcommand{\tlegendre}[2]{\textstyle{\left(\frac{#1}{#2}\right)}}
\newcommand{\psum}{\sideset{}{'}\sum}
\newcommand{\asum}{\sideset{}{^{\ast}}\sum}
\newcommand{\tmod}[1]{\ (\mathrm{mod}\text{ }#1)}
\renewcommand{\bmod}[1]{\ \left(\mathrm{mod}\text{ }#1\right)}
\newcommand{\xto}[1]{\xrightarrow{#1}}
\newcommand{\xfrom}[1]{\xleftarrow{#1}}
\newcommand{\normal}{\mathrel{\unlhd}}
\newcommand{\mf}{\mathfrak}
\newcommand{\mc}{\mathcal}
\newcommand{\ms}{\mathscr}

\newcommand{\Mat}{\mathrm{Mat}}
\newcommand{\GL}{\mathrm{GL}}
\newcommand{\SL}{\mathrm{SL}}
\newcommand{\PSL}{\mathrm{PSL}}
\renewcommand{\O}{\mathrm{O}}
\newcommand{\SO}{\mathrm{SO}}
\newcommand{\U}{\mathrm{U}}
\newcommand{\Sp}{\mathrm{Sp}}

\newcommand{\N}{\mathbb{N}}
\newcommand{\Z}{\mathbb{Z}}
\newcommand{\Q}{\mathbb{Q}}
\newcommand{\R}{\mathbb{R}}
\newcommand{\C}{\mathbb{C}}
\newcommand{\F}{\mathbb{F}}
\renewcommand{\H}{\mathbb{H}}
\renewcommand{\P}{\mathbb{P}}

\renewcommand{\a}{\alpha}
\renewcommand{\b}{\beta}
\newcommand{\g}{\gamma}
\renewcommand{\d}{\delta}
\newcommand{\z}{\zeta}
\renewcommand{\t}{\theta}
\renewcommand{\i}{\iota}
\renewcommand{\k}{\kappa}
\renewcommand{\l}{\lambda}
\newcommand{\s}{\sigma}
\newcommand{\w}{\omega}

\newcommand{\G}{\Gamma}
\newcommand{\D}{\Delta}
\renewcommand{\L}{\Lambda}
\newcommand{\W}{\Omega}
\newcommand{\scL}{\mathscr{L}}

\newcommand{\e}{\varepsilon}
\newcommand{\vt}{\vartheta}
\newcommand{\vphi}{\varphi}
\newcommand{\emt}{\varnothing}

\newcommand{\x}{\times}
\newcommand{\ox}{\otimes}
\newcommand{\op}{\oplus}
\newcommand{\bigox}{\bigotimes}
\newcommand{\bigop}{\bigoplus}
\newcommand{\del}{\partial}
\newcommand{\<}{\langle}
\renewcommand{\>}{\rangle}
\newcommand{\lf}{\lfloor}
\newcommand{\rf}{\rfloor}
\newcommand{\wtilde}{\widetilde}
\newcommand{\what}{\widehat}
\newcommand{\conj}{\overline}
\newcommand{\cchi}{\conj{\chi}}

\DeclareMathOperator{\id}{\textrm{id}}
\DeclareMathOperator{\sgn}{\mathrm{sgn}}
\DeclareMathOperator{\im}{\mathrm{im}}
\DeclareMathOperator{\rk}{\mathrm{rk}}
\DeclareMathOperator{\adj}{\mathrm{adj}}
\DeclareMathOperator{\tr}{\mathrm{trace}}
\DeclareMathOperator{\nm}{\mathrm{norm}}
\DeclareMathOperator{\disc}{\mathrm{disc}}
\DeclareMathOperator{\ord}{\mathrm{ord}}
\DeclareMathOperator{\sym}{\mathrm{sym}}
\DeclareMathOperator{\ext}{\mathrm{ext}}
\DeclareMathOperator{\Hom}{\mathrm{Hom}}
\DeclareMathOperator{\End}{\mathrm{End}}
\DeclareMathOperator{\Aut}{\mathrm{Aut}}
\DeclareMathOperator{\Tor}{\mathrm{Tor}}
\DeclareMathOperator{\Ann}{\mathrm{Ann}}
\DeclareMathOperator{\Gal}{\mathrm{Gal}}
\DeclareMathOperator{\Trace}{\mathrm{Tr}}
\DeclareMathOperator{\Norm}{\mathrm{N}}
\DeclareMathOperator{\Cl}{\mathrm{Cl}}
\DeclareMathOperator{\Span}{\mathrm{Span}}
\DeclareMathOperator*{\Res}{\mathrm{Res}}
\DeclareMathOperator{\Vol}{\mathrm{Vol}}
\DeclareMathOperator{\Li}{\mathrm{Li}}
\DeclareMathOperator{\Supp}{\mathrm{Supp}}
\renewcommand{\Re}{\mathrm{Re}}
\renewcommand{\Im}{\mathrm{Im}}
\DeclareMathOperator{\Ph}{\mathrm{Ph}}
\DeclareMathOperator{\SC}{\mathrm{SC}}


\newcommand{\GH}{\G\backslash\H}
\newcommand{\GG}{\G_{\infty}\backslash\G}

\newenvironment{psmallmatrix}
  {\left(\begin{smallmatrix}}
  {\end{smallmatrix}\right)}

\newcommand{\smc}[1]{
    \mathchoice
    {{\scriptstyle\mathcal{#1}}}
    {{\scriptstyle\mathcal{#1}}}
    {{\scriptscriptstyle\mathcal{#1}}}
    {\scalebox{0.7}{$\scriptscriptstyle\mathcal{#1}$}}
}

%============%
%  Comments  %
%============%
\newcommand{\todo}[1]{\textcolor{red}{\sf Todo: [#1]}}

%===================%
%  Label reminders  %
%===================%
% [label=(\roman*)]
% [label=(\alph*)]
% [label=(\arabic{enumi})]

%==================%
%  Other settings  %
%==================%
\pgfdeclarelayer{background}
\pgfsetlayers{background,main}
\tikzset{->-/.style={decoration={
  markings,
  mark=at position .5 with {\arrow{>}}},postaction={decorate}}}

%=================%
%  Title & Index  %
%=================%
\title{Integral lattices}
\author{Henry Twiss}
\makeindex

\begin{document}
  \date{}
  \maketitle
  \section{Integral Lattices}
    Let \(F\) be a characteristic zero field and \(V\) be an \(n\)-dimensional \(F\)-vector space with nondegenerate symmetric inner product \(\<\cdot,\cdot\>\). We say that a subset \(\L\) of \(V\) is a \textit{lattice} if \(\L\) is a free abelian group. In particular, any lattice \(\L\) is of the form
    \[
      \L = \Z v_{1}+\cdots+\Z v_{m},
    \]
    for some linearly independent vectors \(v_{1},\ldots,v_{m}\) of \(V\) with \(m \le n\). We say \(\L\) is \textit{complete} if \(n = m\). This means \(v_{1},\ldots,v_{n}\) is necessarily a basis of \(V\). If \(e_{1},\ldots,e_{n}\) is an orthonormal basis for \(V\), write
    \[
      v_{i} = \sum_{i}v_{ij}e_{j},
    \]
    with \(v_{ij} \in F\), and define the associated \textit{generator matrix} \(P\) by
    \[
      P = \begin{pmatrix} v_{11} & \cdots & v_{n1} \\ \vdots & & \vdots \\ v_{1n} & \cdots & v_{nn} \end{pmatrix}.
    \]
    Then \(P\) is the base change matrix from \(e_{1},\ldots,e_{n}\) to \(v_{1},\ldots,v_{n}\). The \textit{covolume} \(V_{\L}\) of a complete lattice \(\L\) is defined to be
    \[
      V_{\L} = |\det(P)|.
    \]
    As the base change matrix between any two orthonormal basses is an orthogonal matrix and hence has determinant \(\pm 1\), the covolume is independent of the choice of bases. In the case where \(V\) is a real or complex vector space, we will always take \(e_{1},\ldots,e_{n}\) to be the standard basis.
    
    If \(\L\) is a complete lattice in \(V\) then the \textit{dual} \(\L^{\vee}\) of \(\L\) is defined by
    \[
      \L^{\vee} = \{v \in V:\<\l,v\> \in \Z \text{ for all } \l \in \L\}.
    \]
    In other words, the dual lattice consists of all of the vectors in \(V\) whose inner product with elements of the lattice are integers. Clearly the dual lattice is an abelian group. Less clear is that the dual lattice turns out to always be a complete lattice.

    \begin{proposition}\label{prop:dual_lattice_exists}
      If \(v_{1},\ldots,v_{n}\) is a basis for a complete lattice \(\L\) then the dual basis \(v_{1}^{\vee},\ldots,v_{n}^{\vee}\) is a basis for \(\L^{\vee}\). In particular, \(\L^{\vee}\) is a complete lattice.
    \end{proposition}
    \begin{proof}
      Let \(v \in V\) and write
      \[
        v = \sum_{j}a_{i}v_{j}^{\vee},
      \]
      with \(v_{j}^{\vee} \in \R\). Since the dual basis is determined by \(\<v_{i},v_{j}^{\vee}\> = \d_{i,j}\), it follows that \(v \in \L^{\vee}\) if and only if \(v_{j}^{\vee} \in \Z\). This means \(v_{1}^{\vee},\ldots,v_{n}^{\vee}\) is a basis for \(\L^{\vee}\) as a free abelian group and thus is a complete lattice.
    \end{proof}

    Since the dual of the dual basis is the original basis, \((\L^{\vee})^{\vee} = \L\). We say that \(\L\) is \textit{self-dual} if \(\L^{\vee} = \L\). For example, the lattice \(\Z^{n}\) is self-dual because the standard basis is self-dual. It also turns out that the covolume of the dual lattice is the inverse of the volume of the original lattice:

    \begin{proposition}\label{prop:covolume_of_dual_is_inverse}
      Let \(\L\) be a complete lattice in \(V\). Then
      \[
        V_{\L^{\vee}} = \frac{1}{V_{\L}}.
      \]
    \end{proposition}
    \begin{proof}
      Let \(e_{1},\ldots,e_{n}\) be an orthonormal basis for \(V\), let \(v_{1},\ldots,v_{n}\) be a basis for \(\L\), and let \(P\) be the associated generator matrix. By \cref{prop:dual_lattice_exists}, \(v_{1}^{\vee},\ldots,v_{n}^{\vee}\) is a basis for \(\L^{\vee}\) and \((P^{-1})^{t}\) is the base change matrix from \(e_{1},\ldots,e_{n}\) to \(v_{1}^{\vee},\ldots,v_{n}^{\vee}\). As \(\det((P^{-1})^{t}) = \det(P)^{-1}\) the claim follows.
    \end{proof}
    
    We now turn to the case when \(V\) is an \(n\)-dimensional real inner product space. Let \(d\l\) be the Lebesgue measure induced by the inner product with associated volume \(\Vol\) given by
    \[
      \Vol(M) = \int_{M}\,d\l,
    \]
    for any measurable set \(M\). Note that \(\L\) acts on \(V\) by automorphisms given by translation. In other words, we have a group action
    \[
      \L \x V \to V \qquad (\l,v) \mapsto \l+v.
    \]
    Moreover, \(\L\) acts properly discontinuously on \(V\). To see this, let \(v \in V\) and let \(\d_{v}\) be such that \(0 < \d_{v} < \min_{i}(v-v_{i})\). Taking \(U_{v}\) to be the ball of radius \(\d_{v}\) about \(v\), the intersection \(\l+U_{v} \cap U_{v}\) is empty unless \(\l = 0\). As \(\L\) is also discrete, it follows that \(V/\L\) is also connected Hausdorff. Whence \(V/\L\) admits a fundamental domain
    \[
      \mc{M} = \{t_{1}v_{1}+\cdots+t_{n}v_{n} \in V:0 \le t_{i} \le 1 \text{ for all } i\}.
    \]
    Moreover, any translation of \(\mc{M}\) by an element of \(\L\) is also a fundamental domain. As we might expect, the covolume of \(\L\) is equal to the volume of \(\mc{M}\):

    \begin{proposition}\label{prop:covolume_equals_volume_of_fundamental_domain}
      Let \(\L\) be a complete lattice in a finite dimensional real inner product space \(V\) and let \(\mc{M}\) be a fundamental domain for \(\L\). Then
      \[
        V_{\L} = \Vol(\mc{M}).
      \]
    \end{proposition}
    \begin{proof}
      Let \(e_{1},\ldots,e_{n}\) be an orthonormal basis of \(V\), and \(v_{1},\ldots,v_{n}\) be a basis for \(\L\). Also let \(P\) be the associated generator matrix. The change of variables
      \[
        x_{1}e_{1}+\cdots+x_{n}e_{n} \mapsto x_{1}v_{1}+\cdots+x_{n}v_{n},
      \]
      has Jacobian determinant \(V_{\L}\). Whence
      \[
        \Vol(\mc{M}) = \int_{\mc{M}}\,d\l = V_{\L}\int_{[0,1]^{n}}\,d\mathbf{x} = V_{\L}.
      \]
    \end{proof}

    This result shows that the covolume of a complete lattice in a real inner product space is a measure of the density of the lattice. The smaller the covolume the smaller the fundamental domain and the more dense the lattice is.
    
    It turns out that for a real inner product space, being a lattice is equivalent to being a discrete subgroup:

    \begin{proposition}\label{prop:lattice_if_and_only_if_discrete_subgroup}
      Let \(\L\) be a subset of a finite dimensional real inner product space \(V\). Then \(\L\) is a lattice if and only if it is a discrete subgroup.
    \end{proposition}
    \begin{proof}
      It is clear that if \(\L\) is a lattice then it is a discrete subgroup. So suppose \(\L\) is a discrete subgroup. Then \(\L\) is closed. Let \(V'\) be the subspace spanned by \(\L\) and let its dimension be \(m\). Choosing a basis \(v'_{1},\ldots,v'_{m}\) of \(V'\), set
      \[
        \L' = \Z v'_{1}+\cdots+\Z v'_{m}.
      \]
      Then \(\L'\) is a complete lattice in \(V'\) and \(\L' \subseteq \L \subset V\). We claim \(\L/\L'\) is finite. To see this, let \(\l\) be a representative of a coset in \(\L/\L'\) and let \(\mc{M}'\) be a fundamental domain for \(\L'\) in \(V'\). As \(\mc{M}'\) is a fundamental domain, there exists unique \(w' \in \mc{M}'\) and \(\l' \in \L'\) such that \(\l = w'+\l'\). But then \(w' = \l-\l' \in \mc{M}' \cap \L\) and since \(\mc{M}' \cap \L\) is closed, discrete, and compact, it must be finite. Hence there are finitely many \(w'\) and thus finitely many cosets in \(\L/\L'\). Letting \(|\L/\L'| = q\), we have \(q\L \subseteq \L'\) and therefore
      \[
        \L \subseteq \L' = \Z\frac{1}{q}v'_{1}+\cdots+\Z\frac{1}{q}v'_{m}.
      \]
      In particular, \(\L\) is a subgroup of a free abelian group and therefore is free abelian. This means \(\L\) is a lattice.
    \end{proof}

    As for complete lattices in \(V\), they are equivalent to the existence of a bounded set whose translates cover \(V\):

    \begin{proposition}\label{prop:complete_lattice_if_and_only_if_bounded_translates_cover}
      Let \(\L\) be a lattice in a finite dimensional real inner product space \(V\). Then \(\L\) complete if and only if there exists a bounded subset \(M\) of \(V\) whose translates by \(\L\) cover \(V\).
    \end{proposition}
    \begin{proof}
      First suppose \(\L\) is complete. Then we may take \(M = \mc{M}\) to be a fundamental domain of \(\L\) which is bounded and whose translates by \(\L\) cover \(V\). Now suppose \(\L\) is a lattice and there exists a bounded subset \(M\) of \(V\) whose translates by \(\L\) cover \(V\). Let \(W\) be the subspace of \(V\) spanned by \(\L\). Then \(W\) is closed. Moreover, \(\L\) is complete if and only if \(V = W\) and this is what we will show. So let \(v \in V\). Since the translates of \(M\) by \(\L\) cover \(V\), for every positive integer \(n\) we may write
      \[
        nv = w_{n}+\l_{n},
      \]
      with \(v_{n} \in M\) and \(\l_{n} \in \L\). As \(M\) is bounded, \(\lim_{n \to \infty}\frac{1}{n}v_{n} = 0\). Moreover, \(\frac{1}{n}\l_{n} \in W\) and since \(W\) is closed we must have that \(\lim_{n \to \infty}\frac{1}{n}\l_{n}\) exists and belongs to \(W\). These two limits together imply
      \[
        v = \lim_{n \to \infty}\frac{1}{n}\l_{n},
      \]
      and is an element of \(W\). Thus \(V = W\) which means \(\L\) is complete.
    \end{proof}

    The most important result we will require about lattices is \textit{Minkowski's lattice point theorem} which states that, under some mild conditions, a set of sufficiently large volume in \(V\) contains a nonzero point of a complete lattice:

    \begin{theorem*}[Minkowski's lattice point theorem]
      Suppose \(\L\) is a lattice in an \(n\)-dimensional real inner product space \(V\). Let \(X \subset V\) is a compact convex symmetric set. If
      \[
        \Vol(X) \ge 2^{n}V_{\L},
      \]
      then there exists a nonzero \(\l \in \L \cap X\).
    \end{theorem*}
    \begin{proof}
      We will prove the claim depending on if the inequality is strict or not. First suppose \(\Vol(X) > 2^{n}V_{\L}\). Consider the linear map
      \[
        \phi:\frac{1}{2}X \to V/\L \qquad \frac{1}{2}x \mapsto \frac{1}{2}x+\L.
      \]
      If \(\phi\) were injective then we would have
      \[
        \Vol\left(\frac{1}{2}X\right) = \frac{1}{2^{n}}\Vol(X) \le V_{\L}, 
      \]
      so that \(\Vol(X) \le  2^{n}V_{\L}\). This is a contradiction, so \(\phi\) cannot be injective. Hence there exists distinct \(x_{1},x_{2} \in \frac{1}{2}X\) such that \(\phi(x_{1}) = \phi(x_{2})\). Thus \(2x_{1},2x_{2} \in X\). In particular, since \(X\) is symmetric we must have \(-2x_{2} \in X\). Convexity of \(X\) implies
      \[
        \left(1-\frac{1}{2}\right)2x_{1}+\frac{1}{2}(-2x_{2}) = x_{1}-x_{2},
      \]
      is an element of \(X\). But \(x_{1}-x_{2} \in \L\) because \(\phi(x_{1}) = \phi(x_{2})\) and \(\phi\) is linear. Then \(\l = x_{1}-x_{2}\) is nonzero with \(\l \in \L \cap X\). Now suppose \(\Vol(X) = 2^{n}V_{\L}\). Then
      \[
        \Vol((1+\e)X) = (1+\e)^{n}2^{n}V_{\L} > 2^{n}V_{\L}.
      \]
      What we have just proved shows that there exists a nonzero \(\l_{\e} \in \L \cap (1+\e)X\). In particular, if \(\e \le 1\) then \(\l_{\e} \in 2X \cap \L\). The set \(2X \cap \L\) is compact and discrete and therefore finite. Taking \(\e = \frac{1}{n}\), the sequence \((\l_{\frac{1}{n}})_{n \ge 1}\) belongs to the finite set \(2X \cap \L\) and so must converge to a point \(\l\). Since \(\L\) is discrete and the \(\l_{\frac{1}{n}}\) are nonzero so too is \(\l\). As \(\l\) is an element of
      \[
        \bigcap_{n \ge 1}\left(1+\frac{1}{n}\right)X,
      \]
      and \(X\) is closed, \(\l \in X\) as well. Thus we have found a nonzero \(\l \in \L \cap X\).
    \end{proof}
\end{document}