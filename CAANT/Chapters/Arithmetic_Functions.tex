\chapter{Arithmetic Functions}
  \section{Multiplicative and Additive Functions}
    Any function
    \[
      f:\Z_{\ge 1} \to \C,
    \]
    is said to be \textit{arithmetic}. That is, arithmetic functions are maps from the positive integers to the complex numbers. If an arithmetic function \(f\) satisfies
    \[
      f(nm) = f(n)f(m) \quad \text{or} \quad f(nm) = f(n)+f(m),
    \]
    whenever \(n\) and \(m\) are relatively prime, we say \(f\) is \textit{multiplicative} or \textit{additive} respectively. If either condition holds for all \(n\) and \(m\) then we say \(f\) is \textit{completely multiplicative} or \textit{completely additive}. If \(f\) is additive or multiplicative then \(f\) is uniquely determined by its values on prime powers and if \(f\) is completely additive or completely multiplicative then it is uniquely determined by its values on primes. Moreover, we have
    \[
      f(1) = 1 \quad \text{or} \quad f(1) = 0,
    \]
    according to if \(f\) is multiplicative or additive. Most important arithmetic functions are either multiplicative or additive. Below is a list of the most important arithmetic functions (some of these functions are restrictions of common functions but we define them here as arithmetic functions for completeness):
    \begin{enumerate}[label*=(\roman*)]
      \item The \textit{constant function}: The function \(\mathbf{1}(n)\). This function is neither additive nor multiplicative.
      \item The \textit{indicator function}: The function \(\e(n)\) defined by
      \[
        \e(n) = \begin{cases} 1 & \text{if \(n = 1\)}, \\ 0 & \text{if \(n \ge 2\)}. \end{cases}
      \]
      This function is completely multiplicative.
      \item The \textit{identity function}: The function \(\id(n)\). This function is completely multiplicative.
      \item The \textit{logarithm}: The function \(\log(n)\). This function is completely additive.
      \item The \textit{M\"obius function}: The function \(\mu(n)\) defined by
      \[
        \mu(n) = \begin{cases} 1 & \text{if \(n\) is square-free with an even number of prime factors}, \\ -1 & \text{if \(n\) is square-free with an odd number of prime factors}, \\ 0 & \text{if \(n\) is not square-free}. \end{cases}
      \]
      This function is multiplicative.
      \item The \textit{characteristic function of square-free integers}: The square of the M\"obius function \(\mu^{2}(n)\). This function is multiplicative.
      \item \textit{Liouville's function}: The function \(\l(n)\) defined by
      \[
        \l(n) = \begin{cases} 1 & \text{if \(n = 1\)}, \\ (-1)^{k} & \text{if \(n\) is composed of \(k\) not necessarily distinct prime factors}. \end{cases}
      \]
      This function is completely multiplicative.
      \item \textit{Euler's totient function}: The function \(\vphi(n)\) defined by
      \[
        \vphi(n) = \sum_{\substack{a \tmod{n} \\ (a,n) = 1}}1.
      \]
      This function is multiplicative.
      \item The \textit{divisor function}: The function \(\s_{0}(n)\) defined by
      \[
        \s_{0}(n) = \sum_{d \mid n}1.
      \]
      This function is multiplicative.
      \item The \textit{sum of divisors function}: The function \(\s_{1}(n)\) defined by
      \[
        \s_{1}(n) = \sum_{d \mid n}d.
      \]
      This function is multiplicative.
      \item The \textit{generalized sum of divisors function}: The function \(\s_{s}(n)\) defined by
      \[
        \s_{s}(n) = \sum_{d \mid n}d^{s},
      \]
      for any \(s \in \C\). This function is multiplicative.
      \item The \textit{number of distinct prime factors function}: The function \(\w(n)\) defined by
      \[
        \w(n) = \sum_{p \mid n}1.
      \]
      This function is additive.
      \item The \textit{total number of prime divisors function}: The function \(\W(n)\) defined by
      \[
        \W(n) = \sum_{p ^{m} \mid n}1.
      \]
      This function is completely additive.
      \item The \textit{von Mangoldt function}: The function \(\L(n)\) defined by
      \[
        \L(n) = \begin{cases} 0 & \text{if \(n\) is not a prime power}, \\ \log(p) & \text{if \(n = p^{m}\) for some prime \(p\) and positive integer \(m\)}. \end{cases}
      \]
      This function is neither additive or multiplicative.
    \end{enumerate}
  \section{Dirichlet Convolution and the M\"obius Inversion Formula}
    If \(f\) and \(g\) are two arithmetic functions then we can define a new arithmetic function \(f \ast g\) called their \textit{Dirichlet convolution} defined by
    \[
      (f \ast g)(n) = \sum_{d \mid n}f(d)g\left(\frac{n}{d}\right).
    \]
    As the sum is over all divisors, Dirichlet convolution is a symmetric operation. Dirichlet convolution preserves multiplicative functions.

    \begin{proposition}\label{prop:Dirichlet_convolution_of_multiplicative_functions}
      If \(f\) and \(g\) are multiplicative arithmetic functions then so is their Dirichlet convolution \(f \ast g\).
    \end{proposition}
    \begin{proof}
      Let \(n\) and \(m\) be relatively prime positive integers. Every divisor \(d\) of \(nm\) is of the form \(d = d'd''\) with \(d'\) a divisor of \(n\), \(d''\) a divisor of \(m\), and \(d'\) and \(d''\) relatively prime. A short computation shows
      \[
        \sum_{d \mid nm}f(d)g\left(\frac{nm}{d}\right) = \left(\sum_{d' \mid n}f(d')g\left(\frac{n}{d'}\right)\right)\left(\sum_{d'' \mid m}f(d'')g\left(\frac{m}{d''}\right)\right).
      \]
      This is equivalent to the claim.
    \end{proof}

    This result makes the set of multiplicative functions into a commutative semigroup under Dirichlet convolution. It is actually a commutative monoid since the indicator function \(\e\) acts as an identity. This means
    \[
      f \ast \e = f.
    \]
    A certain case of interest for Dirichlet convolution is when the M\"obius function is convolved with the constant function.

    \begin{proposition}\label{prop:Mobius_indicator}
    We have
    \[
      \sum_{d \mid n}\mu(d) = \begin{cases} 1 & \text{if \(n = 1\)}, \\ 0 & \text{if \(n \ge 2\)}. \end{cases}
    \]
    In particular,
    \[
      \mu \ast \mathbf{1} = \e.
    \]
    \end{proposition}
    \begin{proof}
      The first statement is equivalent to the second, so it suffices to prove the first. The sum \(\sum_{d \mid n}\mu(d)\) is multiplicative by \cref{prop:Dirichlet_convolution_of_multiplicative_functions} so we may assume \(n = p^{r}\) is a nonnegative power of a prime. When \(r = 0\), \(d = 1\) and the sum is \(1\). When \(r \ge 1\), \(d\) runs over \(1,p,\ldots,p^{r}\). Every value is zero except \(\mu(1) = 1\) and \(\mu(p) = -1\). This proves that the sum is zero.
    \end{proof}

    With this result we can prove the infamous \textit{M\"obius inversion formula}:

    \begin{theorem*}[M\"obius inversion formula]
      If \(f\) and \(g\) are arithmetic functions, then
      \[
        g(n) = \sum_{d \mid n}f(d) \quad \text{if and only if} \quad f(n) = \sum_{d \mid n}g(d)\mu\left(\frac{n}{d}\right).
      \]
      In particular,
      \[
        g = f \ast \mathbf{1} \quad \text{if and only if} \quad f = g \ast \mu.
      \]
    \end{theorem*}
    \begin{proof}
      As the first statement is equivalent to the second, we will prove the second statement. Convolving \(g = f \ast \mathbf{1}\) with \(\mu\) gives \(f = g \ast \mu\) in view of \cref{prop:Mobius_indicator}. This proves the forward implication. The reverse implication follows by convolving \(f = g \ast \mu\) with \(\mathbf{1}\) and arguing analogously.
    \end{proof}