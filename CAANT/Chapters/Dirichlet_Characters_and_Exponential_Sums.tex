\chapter{Dirichlet Characters and Exponential Sums}
  \section{Dirichlet Characters}
    Let \(m\) be a positive integer. A multiplicative homomorphism
    \[
      \chi:\Z \to \C,
    \]
    is said to be a \textit{Dirichlet character} modulo \(m\) if it is \(m\)-periodic and such that \(\chi(a) = 0\) if and only if \((a,m) > 1\). We call \(m\) the \textit{modulus} of \(\chi\). A Dirichlet character is necessarily a completely multiplicative arithmetic function when restricted to the positive integers.
    
    We say a Dirichlet character \(\chi\) is \textit{principal} if it only takes values \(0\) or \(1\). There is always a unique principal Dirichlet character modulo \(m\), denoted \(\chi_{m,0}\), defined by
    \[
      \chi_{m,0}(a) = \begin{cases} 1 & (a,m) = 1, \\ 0 & (a,m) > 1. \end{cases}
    \]
    When the modulus is \(1\), the principal Dirichlet character is identically \(1\) and we call this the \textit{trivial Dirichlet character}. This is the only Dirichlet character modulo \(1\).
    
    By Euler's little theorem, \(a^{\vphi(m)} \equiv 1 \tmod{m}\) whenever \((a,m) = 1\). This forces \(\chi(a)^{\vphi(m)} = 1\) and so the nonzero values of any Dirichlet character modulo \(m\) are \(\vphi(m)\)-th roots of unity. This implies that there are only finitely many Dirichlet characters of any fixed modulus. Given two Dirichlet character \(\chi\) and \(\psi\) modulo \(m\), the functions
    \[
      \chi\psi:\Z \to \C \quad \text{and} \quad \cchi:\Z \to \C,
    \]
    are also Dirichlet characters modulo \(m\). This turns the set of such Dirichlet characters into an abelian group denote by \(X_{m}\) where the identity is the principal Dirichlet character modulo \(m\) and the inverse is given by the conjugate as the nonzero values of Dirichlet characters are roots of unity.
    
    This is all strikingly similar to characters on \((\Z/m\Z)^{\ast}\) and there is indeed a connection. By the multiplicativity and \(m\)-periodicity of \(\chi\), it induces a character of \((\Z/m\Z)^{\ast}\). Conversely, if we are given a character on \((\Z/m\Z)^{\ast}\) we can extend it to a Dirichlet character \(\chi\) by defining it to be \(m\)-periodic with \(\chi(a) = 0\) if \((a,m) > 1\). We call this extension a \textit{zero extension}. This argument shows that Dirichlet characters modulo \(m\) are exactly the zero extensions of characters on \((\Z/m\Z)^{\ast}\). As abelian groups are isomorphic to their character groups, we deduce that the group of Dirichlet characters modulo \(m\) is isomorphic to \((\Z/m\Z)^{\ast}\). Therefore there are \(\vphi(m)\) Dirichlet characters modulo \(m\) and we identify them with the characters on \((\Z/m\Z)^{\ast}\). Just as for characters of abelian groups, we have orthogonality relations called the \textit{Dirichlet orthogonality relations}.

    \begin{proposition*}[Dirichlet orthogonality relations]
    Let \(\chi\) and \(\psi\) be Dirichlet characters modulo \(m\) and let \(a,b \in (\Z/m\Z)^{\ast}\). Then
    \[
      \sum_{a \tmod{m}}(\chi\conj{\psi})(a) = \vphi(m)\d_{\chi,\psi} \quad \text{and} \quad \sum_{\chi \tmod{m}}\chi(a\conj{b}) = \vphi(m)\d_{a,b}.
    \]
    In particular,
    \[
      \sum_{a \tmod{m}}\chi(a) = \vphi(m)\d_{\chi,\chi_{m,0}} \quad \text{and} \quad \sum_{\chi \tmod{m}}\chi(a) = \vphi(m)\d_{a,1}.
    \]
    \end{proposition*}
    \begin{proof}
      If \(\chi = \psi\) then the first sum is clearly \(\phi(m)\). If not, let \(b \in (\Z/m\Z)^{\ast}\) be such that \((\chi\conj{\psi})(b) \neq 1\). A short computation shows
      \[
        (\chi\conj{\psi})(b)\sum_{a \tmod{m}}(\chi\conj{\psi})(a) = \sum_{a \tmod{m}}(\chi\conj{\psi})(a),
      \]
      in which case the sum vanishes. This proves the first identity. For the second, if \(a = b\) then the second sum is clearly \(\vphi(m)\). If \(a \neq b\), we claim that there exists a Dirichlet character \(\psi\) modulo \(m\) with \(\psi(a\conj{b}) \neq 1\). Set \(g = a\conj{b}\). The cyclic subgroup \(\<g\>\) of \((\Z/m\Z)^{\ast}\) has some order \(d > 1\). Consider the homomorphism
      \[
        \psi_{d}:\<g\> \to \C \qquad g^{k} \mapsto e^{\frac{2\pi ik}{d}}.
      \]
      By the structure theorem for finite abelian groups, \((\Z/m\Z)^{\ast} \cong \<g\> \x H\) for some subgroup \(H\). Whence we define a Dirichlet character \(\psi\) modulo \(m\) by
      \[
        \psi:(\Z/m\Z)^{\ast} \to \C \qquad g^{k}h \mapsto e^{\frac{2\pi ik}{d}}.
      \]
      This \(\psi\) is the desired Dirichlet character, A short computation shows
      \[
        \psi(a\conj{b})\sum_{\chi \tmod{m}}\chi(a\conj{b}) = \sum_{\chi \tmod{m}}\chi(a\conj{b}),
      \]
      in which case the sum vanishes. This proves the second identity and the first statement in its entirety. The second statement follows from the first upon taking \(\psi = \chi_{m,0}\) and \(b = 1\) respectively.
    \end{proof}

    It it possible for Dirichlet characters of a fixed modulus to arise from Dirichlet characters of a smaller modulus. Suppose \(\chi\) and \(\chi^{\ast}\) are Dirichlet characters modulo \(m\) and \(d\) respectively with \(d \mid m\). We say \(\chi\) is induced \textit{induced} from \(\chi^{\ast}\) (or \(\chi^{\ast}\) \textit{induces} \(\chi\)) if
    \[
      \chi(a) = \begin{cases} \chi^{\ast}(a) & \text{if \((a,m) = 1\)}, \\ 0 & \text{if \((a,m) > 1\)}. \end{cases}
    \] 
    This means \(\chi\) is a Dirichlet character whose values are given by those of \(\chi^{\ast}\). Necessarily \(\chi\) is \(d\)-periodic and its nonzero values are \(\vphi(d)\)-th roots of unity. We say a Dirichlet character is \textit{primitive} if it is not induced by any Dirichlet character other than itself and \textit{imprimitive} otherwise. The principal Dirichlet characters are precisely those induced from the trivial Dirichlet character and the only primitive one is the trivial Dirichlet character itself. Moreover, a Dirichlet character is primitive if and only if its conjugate is. Our primary aim will be to show that every Dirichlet character is induced from a unique primitive Dirichlet character. 
    
    \begin{theorem}\label{thm:Dirichlet_character_conductor_existance}
      Suppose \(\chi\) is a Dirichlet character modulo \(m\). There exists a unique primitive Dirichlet character \(\wtilde{\chi}\) such that \(\chi\) is induced from \(\wtilde{\chi}\).
    \end{theorem}
    \begin{proof}
      Let \(q\) be the positive integer given by
      \[
        q = \min\{d \mid m:\chi(a) = \chi(b) \text{ for all } a \equiv b \tmod{d} \text{ with } (ab,m) = 1\},
      \]
      and let \(\wtilde{\chi}:\Z \to \C\) be defined by
      \[
        \wtilde{\chi}(a) = \begin{cases} \chi(a) & \text{if \((a,q) = 1\)}, \\ 0 & \text{if \((a,q) > 1\)}. \end{cases}
      \]
      The definition of \(q\) implies \(\wtilde{\chi}\) is well-defined and hence \(q\)-periodic. In fact, \(\wtilde{\chi}\) is a Dirichlet character modulo \(q\) and minimality forces \(\wtilde{\chi}\) to be primitive. By construction \(\chi\) is indued from \(\wtilde{\chi}\) and this proves existence. Now suppose \(\wtilde{\chi}_{1}\) and \(\wtilde{\chi}_{2}\) are two primitive Dirichlet characters modulo \(q_{1}\) and \(q_{2}\) respectively both of which induce \(\chi\). Then \(\wtilde{\chi}_{1}(a) = \wtilde{\chi}_{2}(a)\) whenever \((a,m) = 1\). Setting \(q = (q_{1},q_{2})\), we also have \(\wtilde{\chi}_{1}(a) = \wtilde{\chi}_{2}(a)\) whenever \((a,q) = 1\). Hence \(\wtilde{\chi}_{1}\) and \(\wtilde{\chi}_{2}\) are both induced from the same Dirichlet character modulo \(q\). Primitivity implies \(q_{1} = q_{2}\) and \(\wtilde{\chi}_{1} = \wtilde{\chi}_{2}\). This proves uniqueness.
    \end{proof}

    In light of this result, we define the \textit{conductor} \(q\) of a Dirichlet character \(\chi\) modulo \(m\) to be the modulus of the unique primitive Dirichlet character \(\wtilde{\chi}\) inducing \(\chi\). By the proof, the conductor is given by
    \[
      q = \min\{d \mid m:\chi(a) = \chi(b) \text{ for all } a \equiv b \tmod{d} \text{ with } (ab,m) = 1\}.
    \]
    Also observe that \(\chi\) is \(q\)-periodic, \(q\) is the minimal period of \(\chi\), and the nonzero values of \(\chi\) are \(\vphi(q)\)-th roots of unity. Moreover,
    \[
      \chi = \wtilde{\chi}\chi_{\frac{m}{q},0},
    \]
    and \(\chi\) is primitive if and only if its conductor and modulus are equal.
    
    As not every Dirichlet character of a fixed modulus is primitive, it is natural to ask how many primitive Dirichlet characters there are for a given modulus. Let \(N(m)\) be the number of primitive Dirichlet characters modulo \(m\). Then \(N(m)\) is easily determined via M\"obius inversion.

    \begin{proposition}
      For any positive integer \(m\), we have
      \[
        \phi(m) = \sum_{d \mid m}N(d),
      \]
      where \(N(d)\) be the number of primitive Dirichlet characters modulo \(d\). In particular, \(N(m)\) is given by
      \[
        N(m) = \sum_{d \mid m}\phi(d)\mu\left(\frac{m}{d}\right)
      \]
    \end{proposition}
    \begin{proof}
      To prove the first formula, the right-hand side counts the number of Dirichlet characters modulo \(m\) since every such Dirichlet character is induced from a unique primitive Dirichlet character whose modulus divides \(m\) by \cref{thm:Dirichlet_character_conductor_existance}. The left hand side also counts the number of Dirichlet characters modulo \(m\) as are \(\phi(m)\) many. This proves the first formula. The second follows by M\"obius inversion.
    \end{proof}

    Primitive Dirichlet characters also behave well with respect to multiplication if the conductors are relatively prime as the following proposition shows:

    \begin{proposition}\label{prop:primitive_characters_multiplicative_relatively_prime}
      Suppose \(\chi_{1}\) and \(\chi_{2}\) are Dirichlet characters modulo \(m_{1}\) and \(m_{2}\) respectively with \((m_{1},m_{2}) = 1\). Set \(\chi = \chi_{1}\chi_{2}\). Then \(\chi\) is a primitive if and only if \(\chi_{1}\) and \(\chi_{2}\) both are.
    \end{proposition}
    \begin{proof}
      By construction, \(\chi\) is a Dirichlet character modulo \(m_{1}m_{2}\). Let \(q_{1}\) and \(q_{2}\) be the conductors of \(\chi_{1}\) and \(\chi_{2}\) respectively and let \(q\) be the conductor of \(\chi\). Then \(\chi\) is \(q_{1}q_{2}\)-periodic and \(q \mid q_{1}q_{2}\).
      
      For the forward implication, suppose \(\chi\) is primitive so that \(q = m_{1}m_{2}\) whence \(m_{1}m_{2} \mid q_{1}q_{2}\). This forces \(m_{1} = q_{1}\) and \(m_{2} = q_{2}\) proving \(\chi_{1}\) and \(\chi_{2}\) are both primitive. For the reverse implication, suppose \(\chi_{1}\) and \(\chi_{2}\) are both primitive so that \(q_{1} = m_{1}\) and \(q_{2} = m_{2}\). Then \((q_{1},q_{1}) = 1\) and the Chinese remainder theorem implies that \(\chi\) is \(q_{1}\)-periodic on those integers that are congruent to \(1\) modulo \(q_{2}\) and \(q_{2}\)-periodic on those integers that are congruent to \(1\) modulo \(q_{1}\). This forces \(q_{1} \mid q\) and \(q_{2} \mid q\) which together imply \(q = q_{1}q_{2}\) and thus \(\chi\) is primitive.
    \end{proof}
    
    We would now like to distinguish Dirichlet characters based on their nonzero values. We say \(\chi\) is \textit{real} if it is real-valued. This means the nonzero values of \(\chi\) are \(1\) or \(-1\) since they are the only real roots of unity. We say \(\chi\) is \textit{complex} if it is not real. More commonly, we distinguish Dirichlet characters by the roots of unity that their nonzero values take. We say \(\chi\) is of \textit{order} \(n\) if the nonzero values of \(\chi\) are all \(n\)-th roots of unity. In the cases of small order we will often use the latin derived names \textit{quadratic}, \textit{cubic}, etc. To connect these two naming conventions observe that a nontrivial Dirichlet character is quadratic if and only if it is real and an other nontrivial Dirichlet character is necessarily complex. 
    
    We will also distinguish Dirichlet characters by their parity. By multiplicativity, we must have \(\chi(-1) = \pm 1\). Accordingly, we say \(\chi\) is \textit{even} if \(\chi(-1) = 1\) and \textit{odd} if \(\chi(-1) = -1\). Then even Dirichlet characters are even functions while odd Dirichlet characters are odd functions. Note that conjugate and induced Dirichlet characters necessarily have the same parity. The parity is also expressed via the formula
    \[
      \frac{\chi(1)-\chi(-1)}{2} = \begin{cases} 0 & \text{if \(\chi\) is even}, \\ 1 & \text{if \(\chi\) is odd}. \end{cases}
    \]
  \section{Quadratic Dirichlet Characters}
    Quadratic Dirichlet characters deserve special attention as it is possible to classify all of them explicitly. This is due to the fact that they arise from Jacobi and Kronecker symbols. For a positive odd integer \(m\), define
    \[
      \chi_{m}(a) = \tlegendre{a}{m}.
    \]
    By definition of the Jacobi symbol, \(\chi_{m}\) becomes a quadratic Dirichlet character modulo \(m\). Unfortunately, the quadratic Dirichlet characters constructed in this manner do not exhaust all possible examples. To accomplish this we need to use Kronecker symbols. An integer \(D\) is said to be a \textit{fundamental discriminant} if it is of the form
    \[
      D = \begin{cases} d & \text{if \(D \equiv 1 \tmod{4}\)}, \\ 4d & \text{if \(D \equiv 8,12 \tmod{16}\)}, \end{cases}
    \]
    for some square-free integer \(d\). Necessarily \(d \equiv 1 \tmod{4}\) or \(d \equiv 2,3 \tmod{4}\) respectively and thus is nonzero. We define \(\chi_{D}:\Z \to \C\) by
    \[
      \chi_{D}(a) = \legendre{D}{a}.
    \]
    It turns out that \(\chi_{D}\) defines a primitive quadratic Dirichlet character modulo \(|D|\), provided \(D \neq 1\), and exhausts all such possibilities.

    \begin{theorem}\label{thm:fundamental_discriminant_character_primitive}
      If \(D\) is a fundamental discriminant and \(D \neq 1\) then \(\chi_{D}\) is a primitive quadratic Dirichlet character of conductor \(|D|\). Moreover, all primitive quadratic Dirichlet characters are of this form.
    \end{theorem}
    \begin{proof}
      We first show that \(\chi_{D}\) is a primitive quadratic Dirichlet character modulo \(|D|\). If \(D \equiv 1 \tmod{4}\), the sign in quadratic reciprocity is always \(1\). Then
      \[
        \chi_{D}(a) = \legendre{a}{|D|},
      \]
      which is a Dirichlet character modulo \(|D|\). If \(D \equiv 12 \tmod{16}\) then \(\frac{D}{4} \equiv 3 \tmod{4}\) and the sign in quadratic reciprocity is \(\tlegendre{-1}{a}\) which is the primitive quadratic Dirichlet character modulo \(4\) as there are only two such Dirichlet characters and \(\legendre{-1}{a}\) is not principal. Whence
      \[
        \chi_{D}(a) = \legendre{-1}{a}\legendre{a}{\left|\frac{D}{4}\right|},
      \]
      which is a Dirichlet character modulo \(|D|\). If \(D \equiv 8 \tmod{16}\) first observe that \(\tlegendre{D}{a} = \tlegendre{8}{a}\tlegendre{\frac{D}{8}}{a}\) where \(\tlegendre{8}{a}\) is one of the two primitive quadratic Dirichlet character modulo \(8\) (the other is \(\tlegendre{-8}{a}\)). As \(\frac{D}{8} \equiv 1,3 \tmod{4}\), the sign in quadratic reciprocity is either \(1\) or \(\tlegendre{-1}{a}\) according to these two cases. Thus
      \[
        \chi_{D}(a) = \legendre{8}{a}\legendre{a}{\left|\frac{D}{8}\right|} \quad \text{or} \quad \chi_{D}(a) = \legendre{-8}{a}\legendre{a}{\left|\frac{D}{8}\right|},
      \]
      according to if \(\frac{D}{8} \equiv 1,3 \tmod{4}\) respectively, and in either case is a Dirichlet character modulo \(|D|\). We can compactly express all of these cases as follows:
      \[
        \chi_{D}(a) = \begin{cases} \legendre{a}{|D|} & \text{if \(D \equiv 1 \tmod{4}\)}, \\ \legendre{-1}{a}\legendre{a}{\left|\frac{D}{4}\right|} & \text{if \(\frac{D}{4} \equiv 3 \tmod{4}\)}, \\ \legendre{8}{a}\legendre{a}{\left|\frac{D}{8}\right|} & \text{if \(\frac{D}{8} \equiv 1 \tmod{4}\)}, \\ \legendre{-8}{a}\legendre{a}{\left|\frac{D}{8}\right|} & \text{if \(\frac{D}{8} \equiv 3 \tmod{4}\)}. \end{cases}
      \]
      This proves \(\chi_{D}\) is a Dirichlet characters modulo \(|D|\). It is not hard to see that \(\chi_{D}\) is primitive. Indeed, we have already mentioned that the characters \(\tlegendre{-1}{a}\), \(\tlegendre{8}{a}\), and \(\tlegendre{-8}{a}\) are all primitive. Since \(D\), \(\frac{D}{4}\), and \(\frac{D}{8}\) are square-free according to their equivalences modulo \(4\) and \(D \neq 1\), it suffices to show that \(\chi_{p}\) is primitive for all primes \(p\) with \(p \neq 2\) by \cref{prop:primitive_characters_multiplicative_relatively_prime}. This is immediate since \(p\) is prime and \(\chi_{p}\) is not principal. 
      
      We now show that every primitive quadratic Dirichlet character is of the form \(\chi_{D}\) for some fundamental discriminant \(D\). By \cref{prop:primitive_characters_multiplicative_relatively_prime} again, it suffices to consider primitive quadratic Dirichlet character modulo a prime power \(p^{m}\).
      
      First suppose \(p \neq 2\). Then \((\Z/p^{m}\Z)^{\ast}\) is cyclic, generated by say \(g\), and every \(a \in (\Z/p^{m}\Z)^{\ast}\) is of the form \(a = g^{\nu}\) for some \(\nu \in (\Z/\vphi(p^{m})\Z)\). It follows that every corresponding Dirichlet character \(\chi\) is of the form
      \[
        \chi(a) = e^{\frac{2\pi ik\nu}{\vphi(p^{m})}},
      \]
      for an integer \(k\) modulo \(\vphi(p^{m})\). Moreover, \(\chi\) is primitive if and only if \(k \not\equiv 0 \tmod{p}\) for otherwise \(\chi\) is a Dirichlet character modulo \(p^{m-1}\). Similarly, \(\chi\) is quadratic if and only if \(k \equiv \frac{\vphi(p^{m})}{2} \tmod{\vphi(p^{m})}\). Such a \(k\) exists and is unique because \(p \neq 2\). We also see that if \(\chi\) is quadratic then it is imprimitive unless \(m = 1\) for then \(\vphi(p) = p-1\) which is not a multiple of \(p\). To summarized, there is a unique quadratic Dirichlet character modulo \(p^{m}\) and it is primitive if and only if \(m = 1\). Necessarily, this unique primitive quadratic Dirichlet character modulo \(p\) is given by \(\chi_{D}\) for the fundamental discriminant \(D = p\) if \(p \equiv 1 \tmod{4}\) and \(D = -p\) if \(p \equiv 3 \tmod{4}\).
      
      Now suppose \(p = 2\) so that \(p^{m} = 2^{m}\). If \(m = 1\) then \(\vphi(2) = 1\) and there are no primitive quadratic Dirichlet characters as the only Dirichlet character is principal. If \(m = 2\) then \(\vphi(4) = 2\) so that there are two Dirichlet characters. They are both quadratic but only one is primitive namely the aforementioned \(\tlegendre{-1}{a}\). This primitive quadratic Dirichlet character is given by \(\chi_{D}\) for the fundamental discriminant \(D = -4\). For \(m \ge 3\) there is an isomorphism \((\Z/2^{m}\Z)^{\ast} \cong C_{2} \x C_{2^{m-2}}\) where \(C_{2}\) and \(C_{2^{m-2}}\) are the cyclic groups of order \(2\) and \(2^{m-2}\) respectively. Therefore every \(a \in (\Z/2^{m}\Z)^{\ast}\) is of the form \(a = (-1)^{\mu}5^{\nu}\) for some \(\mu \in \Z/2\Z\) and \(\nu \in \Z/2^{m-2}\Z\). Then every corresponding Dirichlet character \(\chi\) is of the form
      \[
        \chi(a) = e^{\frac{2\pi ij\mu}{2}}e^{\frac{2\pi ik\nu}{2^{m-2}}},
      \]
      for integers \(j\) modulo \(2\) and \(k\) modulo \(2^{m-2}\). Similarly to the case for \(p \neq 2\), \(\chi\) is primitive if and only if \(k \not\equiv 0 \tmod{2^{m-2}}\). Moreover, \(\chi\) is quadratic if and only if \(k \equiv 0 \tmod{2^{m-3}}\). These congruences together imply that a primitive quadratic Dirichlet character exists if and only if \(m = 3\). In this case there are four Dirichlet characters. They are all quadratic but only two are primitive, namely the aforementioned \(\tlegendre{8}{a}\) and \(\tlegendre{-8}{a}\). These two primitive quadratic Dirichlet characters are given by \(\chi_{D}\) for the fundamental discriminants \(D = 8\) and \(D = -8\) respectively.
    \end{proof}

    It follows from \cref{thm:fundamental_discriminant_character_primitive} that all quadratic Dirichlet characters are induced from some \(\chi_{D}\) including \(D = 1\) since this corresponds to the trivial Dirichlet character. In particular, so too are the quadratic Dirichlet characters given by Jacobi symbols.
  \section{Ramanujan and Gauss Sums}
    For integers \(b\) and \(m\) with \(m\) positive, the \textit{Ramanujan sum} \(r(b,m)\) is defined by
    \[
      r(b,m) = \sum_{\substack{a \tmod{m} \\ (a,m) = 1}}e^{\frac{2\pi iab}{m}}.
    \]
    The Ramanujan sum is a finite sum of \(m\)-th roots of unity. When \(b = 0\) the Ramanujan sum has the simple evaluation \(r(0,m) = \vphi(m)\). For a general idex the Ramanujan sums can be computed explicitly by means of the M\"obius function.

    \begin{proposition}\label{prop:Ramanujan_sum_evaluation}
      For integers \(b\) and \(m\) with \(m\) positive, we have
      \[
        r(b,m) = \sum_{d \mid (b,m)}d\mu\left(\frac{m}{d}\right).
      \]
    \end{proposition}
    \begin{proof}
      Every \(a\) modulo \(m\) is of the form \(a = a'd\) for some divisor \(d\) of \(m\) and \(a'\) modulo \(\frac{m}{d}\) with \(\left(a',\frac{m}{d}\right) = 1\). So summing \(r(b,d)\) over the divisors \(d\) of \(m\) gives
      \[
        \sum_{d \mid m}r(b,d) = \sum_{a \tmod{m}}e^{\frac{2\pi iab}{m}},
      \]
      If \(m \mid b\) the latter sum is \(m\) while if \(m \nmid b\) the sum vanishes as it is the sum of all \(m\)-th roots of unity. Thus
      \[
        \sum_{d \mid m}r(b,d) = \begin{cases} m & \text{if \(m \mid b\)}, \\ 0 & \text{if \(m \nmid b\)}. \end{cases}
      \]
      Now apply M\"obius inversion.
    \end{proof}

    More general Ramanujan sums can be constructed by introducing a Dirichlet character. Let \(\chi\) be a Dirichlet character modulo \(m\). For any integer \(b\), the \textit{Ramanujan sum} \(\tau(n,\chi)\) associated to \(\chi\) is given by
    \[
      \tau(b,\chi) = \sum_{a \tmod{m}}\chi(a)e^{\frac{2\pi iab}{m}}.
    \]
    This generalizes the previous Ramanujan sum as
    \[
      r(b,m) = \tau(b,\chi_{m,0}).
    \]
    When \(b = 0\), the Dirichlet orthogonality relations imply
    \[
      \tau(0,\chi) = \vphi(m)\d_{\chi,\chi_{m,0}}.
    \]
    When \(b = 1\) we simply write \(\tau(\chi) = \tau(1,\chi)\) and call \(\tau(\chi)\) the \textit{Gauss sum} associated to \(\chi\). The following proposition develops the basic properties of these Ramanujan sums:

    \begin{proposition}\label{prop:Gauss_sum_reduction}
      Let \(\chi\) and \(\psi\) be nontrivial Dirichlet characters modulo \(m\) and \(n\) respectively and let \(b\) be an integer. Then the following hold:
      \begin{enumerate}[label*=(\roman*)]
        \item \(\conj{\tau(b,\cchi)} = \chi(-1)\tau(b,\chi)\).
        \item If \((b,m) = 1\) then \(\tau(b,\chi) = \cchi(b)\tau(\chi)\).
        \item If \((b,m) > 1\) and \(\chi\) is primitive then \(\tau(b,\chi) = 0\).
        \item If \((m,n) = 1\) then \(\tau(b,\chi\psi) = \chi(n)\psi(m)\tau(b,\chi)\tau(b,\psi)\).
        \item If \(\wtilde{\chi}\) is the primitive Dirichlet character of conductor \(q\) inducing \(\chi\), then
        \[
          \tau(\chi) = \mu\left(\frac{m}{q}\right)\wtilde{\chi}\left(\frac{m}{q}\right)\tau(\wtilde{\chi}).
        \]
      \end{enumerate}
    \end{proposition}
    \begin{proof}
      We will prove the properties separately.
      \begin{enumerate}[label*=(\roman*)]
        \item This follows by direct computation.
        \item This follows by direct computation.
        \item Let \(c\) be an integer with \((c,m) = 1\) and satisfying \(\chi(c) \neq 1\). Such a \(c\) exists because otherwise \(\chi\) the principal Dirichlet character modulo \(m\) and thus imprimitive. A short computation shows
        \[
          \chi(c)\tau(b,\chi) = \tau(b,\chi),
        \]
        whence \(\tau(b,\chi) = 0\).
        \item Since \((m,n) = 1\), the Chinese remainder theorem implies that any \(a\) modulo \(mn\) is of the form \(a = a'n+a''m\) with \(a'\) modulo \(m\) and \(a''\) modulo \(n\). Whence
        \[
          (\chi\psi)(a) = \chi(a'n)\psi(a''m).
        \]
        Using this fact, a short computation shows
        \begin{align*}
          \sum_{a \tmod{mn}}(\chi\psi)(a)e^{\frac{2\pi iab}{mn}} &= \chi(n)\psi(m)\\
          &\cdot \left(\sum_{a' \tmod{m}}\chi(a')e^{\frac{2\pi iab}{m}}\right)\left(\sum_{a'' \tmod{n}}\psi(a'')e^{\frac{2\pi ia''b}{n}}\right),
        \end{align*}
        which is equivalent to the claim.
        \item First consider the case when \(\left(\frac{m}{q},q\right) = 1\). In view of \(\chi = \wtilde{\chi}\chi_{\frac{m}{q},0}\), we use (iv) to obtain
        \[
          \tau(\chi) = \tau(\chi_{\frac{m}{q},0})\wtilde{\chi}\left(\frac{m}{q}\right)\tau(\wtilde{\chi}).
        \]
        As \(\tau(\chi_{\frac{m}{q},0}) = r\left(1,\frac{m}{q}\right)\), we use \cref{prop:Ramanujan_sum_evaluation} to compute \(\tau(\chi_{\frac{m}{q},0}) = \mu\left(\frac{m}{q}\right)\). Whence
        \[
          \tau(\chi) = \mu\left(\frac{m}{q}\right)\wtilde{\chi}\left(\frac{m}{q}\right)\tau(\wtilde{\chi}).
        \]
        Now suppose \(\left(\frac{m}{q},q\right) > 1\). In this case the right-hand side is zero because \(\wtilde{\chi}\left(\frac{m}{q}\right) = 0\). So we must show \(\tau(\chi) = 0\). Now there exists a prime \(p\) with \(p \mid \frac{m}{q}\) and \(p \mid q\). For any \(a\) modulo \(m\) write \(a = a'\frac{m}{p}+a''\) with \(a'\) modulo \(p\) and \(a''\) modulo \(\frac{m}{p}\). Moreover, as \(p \mid \frac{m}{q}\) we know \(q \mid \frac{m}{p}\). These two facts and a short computation together show
        \[
          \tau(\chi) = \left(\sum_{a' \tmod{p}}e^{\frac{2\pi ia'}{p}}\right)\left(\sum_{a'' \tmod{\frac{m}{p}}}\wtilde{\chi}(a'')e^{\frac{2\pi ia''}{m}}\right).
        \] 
        The first sum vanishes since it is the sum of all \(p\)-th roots of unity. This proves \(\tau(\chi) = 0\).
      \end{enumerate}
    \end{proof}

    These properties reduce the evaluation of the Ramanujan sum \(\tau(b,\chi)\) to that of the Gauss sum \(\tau(\chi)\) at least when \(\chi\) is primitive. When \(\chi\) is imprimitive and \((b,m) > 1\) we need to appeal to evaluating the Ramanujan sum by more direct means. In fact, even evaluating the Gauss sum for arbitrary primitive Dirichlet characters is a very difficult problem much of which is still open. However, it is not difficult to determine the modulus of the Gauss sum when \(\chi\) is primitive.

    \begin{theorem}\label{thm:Gauss_sum_modulus}
      Let \(\chi\) be a primitive Dirichlet character of conductor \(q\). Then
      \[
        |\tau(\chi)| = \sqrt{q}.
      \]
    \end{theorem}
    \begin{proof}
      If \(\chi\) is the trivial Dirichlet character this is obvious since \(\tau(\chi) = 1\). So assume \(\chi\) is nontrivial whence \(q > 1\). Consider instead \(|\tau(\chi)|^{2} = \tau(\chi)\conj{\tau(\chi)}\). Expanding the Gauss sum \(\conj{\tau(\chi)}\) and invoking \cref{prop:Gauss_sum_reduction} (ii), a short computation shows
      \[
        |\tau(\chi)|^{2} = \sum_{a \tmod{q}}\tau(a,\chi)e^{-{\frac{2\pi ia}{q}}}.
      \]
      Upon expanding the Ramanujan sum, another short computation gives
      \[
        |\tau(\chi)|^{2} = \sum_{a' \tmod{q}}\chi(a')\left(\sum_{a \tmod{q}}e^{\frac{2\pi ia(a'-1)}{q}}\right).
      \]
      If \(a' \equiv 1 \tmod{q}\) the inner sum is \(q\) and otherwise vanishes as it is the sum of all \(q\)-th roots of unity. It follows that the double sum is \(q\) whence \(|\tau(\chi)|^{2} = q\). This is equivalent to the claim.
    \end{proof}

    As an almost immediate corollary, we deduce a useful expression for primitive Dirichlet characters as exponential sums.

    \begin{corollary}\label{cor:gauss_sum_primitive_formula}
      Let \(\chi\) be a primitive Dirichlet character of conductor \(q\). Then for any integer \(b\), we have
      \[
        \tau(b,\chi) = \cchi(b)\tau(\chi).
      \]
      In particular,
      \[
        \chi(b) = \frac{1}{\tau(\cchi)}\sum_{a \tmod{q}}\cchi(a)e^{\frac{2\pi iab}{q}}.
      \]
    \end{corollary}
    \begin{proof}
      For the first statement, the identity is obvious if \(\chi\) is the trivial character as \(\tau(n,\chi) = 1\). So assume \(\chi\) is nontrivial. If \((b,q) = 1\) then this is exactly \cref{prop:Gauss_sum_reduction} (ii). If \((b,q) > 1\) then the identity follows from \cref{prop:Gauss_sum_reduction} (iii) and that \(\cchi(b) = 0\). This proves the first identity in full. For the second identity, observe that \(\tau(\chi) \neq 0\) by \cref{thm:Gauss_sum_modulus}. The second identity follows upon replacing \(\chi\) with \(\cchi\), dividing by \(\tau(\chi)\), and expanding the Ramanujan sum.
    \end{proof}

    For a Dirichlet character \(\chi\) modulo \(m\), we define its \textit{epsilon factor} \(\e_{\chi}\) by
    \[
      \e_{\chi} = \frac{\tau(\chi)}{\sqrt{m}}.
    \]
    When \(\chi\) is primitive, the epsilon factor lies on the unit circle by \cref{thm:Gauss_sum_modulus}. The question of the evaluation of Gauss sums, and hence Ramanujan sums, boils down to determining what value the epsilon factor is. This is the real difficultly in evaluating Gauss sums. However, when \(\chi\) is primitive there is a simple relationship between the epsilon factors \(\e_{\chi}\) and \(\e_{\cchi}\):

    \begin{proposition}\label{prop:epsilon_factor_relationship}
      Let \(\chi\) be a primitive Dirichlet character of conductor \(q\). Then
      \[
        \e_{\chi}\e_{\cchi} = \chi(-1).
      \]
    \end{proposition}
    \begin{proof}
      If \(\chi\) is trivial this is obvious since \(\e_{\chi} = \e_{\cchi} = 1\). So assume \(\chi\) is nontrivial. On the one hand, \(\e_{\cchi}\) lies on the unit circle so that
      \[
        \e_{\cchi}^{-1} = \frac{\conj{\tau(\chi)}}{{\sqrt{q}}}.
      \]
      On the other hand, \cref{prop:Gauss_sum_reduction} (i) implies
      \[
        \e_{\chi} = \chi(-1)\conj{\frac{\tau(\chi)}{\sqrt{q}}}.
      \]
      Combining these identities gives the result.
    \end{proof}
  \section{\todo{Quadratic Gauss Sums}}
    Another important sum is the quadratic Gauss sum. For any \(m \ge 1\) and any \(b \in \Z\), the \textit{quadratic Gauss sum} \(g(b,m)\) is defined by
    \[
      g(b,m) = \sum_{a \tmod{m}}e^{\frac{2\pi ia^{2}b}{m}}.
    \]
    If \(b = 1\) we write \(g(m)\) instead. That is, \(g(m) = g(1,m)\). It turns out that if \(\chi_{m}\) is the quadratic Dirichlet character given by the Jacobi symbol then \(\tau(b,\chi_{m}) = g(b,m)\) provided \(m\) is square-free. This will take a little work to prove. We first reduce to the case when \((b,m) = 1\):

    \begin{proposition}\label{prop:quadratic_Gauss_sum_relatively_prime_reduction}
      Let \(m \ge 1\) be odd and let \(b \in \Z\). Then
      \[
        g(b,m) = (b,m)g\left(\frac{b}{(b,m)},\frac{m}{(b,m)}\right).
      \]
    \end{proposition}
    \begin{proof}
      By Euclidean division write any \(a\) modulo \(m\) in the form \(a = a'\frac{m}{(b,m)}+a''\) with \(a'\) take modulo \((b,m)\) and \(a''\) take modulo \(\frac{m}{(b,m)}\). Then
      \begin{align*}
        g(b,m) &= \sum_{a \tmod{m}}e^{\frac{2\pi ia^{2}b}{m}} \\
        &= \sum_{\substack{a' \tmod{(b,m)} \\ a'' \tmod{\frac{m}{(b,m)}}}}e^{\frac{2\pi i\left(a'\frac{m}{(b,m)}+a''\right)^{2}b}{m}} \\
        &= \sum_{a'' \tmod{\frac{m}{(b,m)}}}e^{\frac{2\pi i(a'')^{2}b}{m}}\sum_{a' \tmod{(b,m)}}e^{\frac{2\pi i\left(2a''a'\frac{m}{(b,m)}+\left(a'\frac{m}{(b,m)}\right)^{2}\right)b}{m}} \\
        &= \sum_{a'' \tmod{\frac{m}{(b,m)}}}e^{\frac{2\pi i(a'')^{2}\frac{b}{(b,m)}}{\frac{m}{(b,m)}}}\sum_{a' \tmod{(b,m)}}e^{\frac{2\pi i\left(2a''a'\frac{m}{(b,m)}+\left(a'\frac{m}{(b,m)}\right)^{2}\right)\frac{b}{(b,m)}}{\frac{m}{(b,m)}}} \\
        &= (b,m)\sum_{a'' \tmod{\frac{m}{(b,m)}}}e^{\frac{2\pi i(a'')^{2}\frac{b}{(b,m)}}{\frac{m}{(b,m)}}},
      \end{align*}
      where the last line follows because \(\left(2a''a'\frac{m}{(b,m)}+\left(a'\frac{m}{(b,m)}\right)^{2}\right) \equiv 0 \tmod{\frac{m}{(b,m)}}\) and thus the inner sum is \((b,m)\). The remaining sum is \(g\left(\frac{b}{(b,m)},\frac{m}{(b,m)}\right)\) which finishes the proof.
    \end{proof}

    As a consequence of \cref{prop:quadratic_Gauss_sum_relatively_prime_reduction}, we may always assume \((b,m) = 1\). Now we give an equivalent formulation of the Ramanujan sum associated to quadratic Dirichlet characters given by Jacobi symbols and show that in the case \(m = p\) an odd prime, the Ramanujan and quadratic Gauss sums agree:

    \begin{proposition}\label{prop:Gauss_sum_equivalence_for_primes}
      Let \(m \ge 1\) and \(b \in \Z\) be such that \((b,m) = 1\). Also let \(\chi_{m}\) be the quadratic Dirichlet character given by the Jacobi symbol. Then
      \[
        \tau(b,\chi_{m}) = \sum_{a \tmod{m}}\left(1+\legendre{a}{m}\right)e^{\frac{2\pi iab}{m}}.
      \]
      Moreover, when \(m = p\) is prime,
      \[
        \tau(b,\chi_{p}) = g(b,p).
      \]
    \end{proposition}
    \begin{proof}
      If \(m = 1\) the claim is obvious since \(\tau(b,\chi_{1}) = 1\) so assume \(m > 1\). To prove the first statement, observe that
      \[
        \sum_{a \tmod{m}}\left(1+\legendre{a}{m}\right)e^{\frac{2\pi iab}{m}} = \sum_{a \tmod{m}}e^{\frac{2\pi iab}{m}}+\sum_{a \tmod{m}}\legendre{a}{m}e^{\frac{2\pi iab}{m}}.
      \]
      The first sum on the right-hand side is zero as it is the sum over all \(m\)-th roots of unity since \((b,m) = 1\). This proves the first claim. Now let \(m = p\) be an odd prime. From the definition of the Jacobi symbol we see that \(1+\tlegendre{a}{p} = 2,0\) depending on if \(a\) is a quadratic residue modulo \(p\) or not provided \(a \not\equiv 0 \tmod{p}\). If \(a \equiv 0 \tmod{p}\) then \(1+\tlegendre{a}{p} = 1\). Moreover, if \(a\) is a quadratic residue modulo \(p\) then \(a \equiv (a')^{2} \tmod{p}\) for some \(a'\). So one the one hand,
      \[
        \tau(b,\chi_{p}) = \sum_{a \tmod{p}}\left(1+\legendre{a}{p}\right)e^{\frac{2\pi iab}{p}} = 1+2\sum_{\substack{a \tmod{p} \\ a \equiv (a')^{2} \tmod{p} \\ a \not\equiv 0 \tmod{p}}}e^{\frac{2\pi i(a')^{2}b}{p}}.
      \]
      On the other hand,
      \[
        g(b,p) = 1+\sum_{\substack{a \tmod{p} \\ a \not\equiv 0 \tmod{p}}}e^{\frac{2\pi ia^{2}b}{p}},
      \]
      but this last sum counts every quadratic residue twice because \((-a)^{2} = a^{2}\). Hence the previous two sums are equal completing the proof.
    \end{proof}

    We would like to generalize the second statement in \cref{prop:Gauss_sum_equivalence_for_primes} to when \(m\) is square-free. In this direction, a series of reduction properties will be helpful:

    \begin{proposition}\label{prop:quadratic_Gauss_sum_reduction}
      Let \(m,n \ge 1\), \(p\) be an odd prime, and \(b \in \Z\). Then the following hold:
      \begin{enumerate}[label*=(\roman*)]
        \item If \((b,p) = 1\) then \(g(b,p^{r}) = pg(b,p^{r-2})\) for all \(r \in \Z\) with \(r \ge 2\).
        \item If \((m,n) = 1\) and \((b,mn) = 1\) then \(g(b,mn) = g(bn,m)g(bm,n)\).
        \item If \(m\) is odd and \((b,m) = 1\) then \(g(b,m) = \tlegendre{b}{m}g(m)\) where \(\tlegendre{b}{m}\) is the Jacobi symbol.
      \end{enumerate}
    \end{proposition}
    \begin{proof}
      We will prove the statements separately.
      \begin{enumerate}[label*=(\roman*)]
        \item First notice that
        \[
          g(b,p^{r}) = \sum_{a \tmod{p^{r}}}e^{\frac{2\pi ia^{2}b}{p^{r}}} = \psum_{a \tmod{p^{r}}}e^{\frac{2\pi ia^{2}b}{p^{r}}}+\sum_{a \tmod{p^{r-1}}}e^{\frac{2\pi ia^{2}b}{p^{r-2}}},
        \]
        since every \(a\) modulo \(p\) satisfies \((a,p) = 1\) or not. By Euclidean division every element \(a\) modulo \(p^{r-1}\) is of the form \(a = a'p^{r-2}+a''\) with \(a'\) taken modulo \(p\) and \(a''\) taken modulo \(p^{r-2}\). Since \((a'p^{r-2}+a'') \equiv a'' \tmod{p^{r-2}}\), every \(a''\) is counted \(p\) times modulo \(p^{r-2}\). Along with the fact that \((a'p^{r-2}+a'')^{2} \equiv (a'')^{2} \tmod{p^{r-2}}\), we have
        \[
          \sum_{a \tmod{p^{r-1}}}e^{\frac{2\pi ia^{2}b}{p^{r-2}}} = \sum_{\substack{a' \tmod{p} \\ a'' \tmod{p^{r-2}}}}e^{\frac{2\pi i\left(a'p^{r-2}+a''\right)^{2}b}{p^{r-2}}} = p\sum_{a'' \tmod{p}}e^{\frac{2\pi i(a'')^{2}b}{p^{r-2}}} = pg(b,p^{r-2}).
        \]
        It remains to show that the sum
        \[
          \psum_{a \tmod{p^{r}}}e^{\frac{2\pi ia^{2}b}{p^{r}}},
        \]
        is zero. As this sum is exactly \(r(b,p^{r})\), \cref{prop:Ramanujan_sum_evaluation} implies
        \[
          \psum_{a \tmod{p^{r}}}e^{\frac{2\pi ia^{2}b}{p^{r}}} = \mu(p^{r}) = 0,
        \]
        because \((b,p) = 1\) and \(r \ge 2\). This proves (i).
        \item Observe that
          \[
            g(bn,m)g(bm,n) = \left(\sum_{a \tmod{m}}e^{\frac{2\pi ia^{2}bn}{m}}\right)\left(\sum_{a' \tmod{n}}e^{\frac{2\pi i(a')^{2}bm}{n}}\right) = \sum_{\substack{a \tmod{m} \\ a' \tmod{n}}}e^{\frac{2\pi i\left((an)^{2}+(a'm)^{2}\right)b}{mn}}.
          \]
          Since \((m,n) = 1\), the Chinese remainder theorem gives an isomorphism
          \[
            (\Z/m\Z) \op (\Z/n\Z) \to (\Z/mn\Z) \qquad a \oplus a' \mapsto an+a'm.
          \]
          Set \(a'' = an+a'm\) so that \((a'')^{2} \equiv (an)^{2}+(a'm)^{2} \tmod{mn}\). Under this isomorphism, the last sum above is then equal to
          \[
            \sum_{a'' \tmod{mn}}e^{\frac{2\pi i(a'')^{2}b}{mn}},
          \]
          which is precisely \(g(b,mn)\). This proves (ii).
        \item The claim is obvious if \(m = 1\) because \(g(b,1) = 1\) so assume \(m > 1\). If \(m = p\) then \cref{prop:Gauss_sum_equivalence_for_primes}, \cref{prop:Gauss_sum_reduction} (ii), and that quadratic Dirichlet characters are their own conjugate altogether imply the claim. Now let \(r \ge 1\) and assume by induction that the claim holds when \(m = p^{r'}\) for all positive integers \(r'\) such that \(r' < r\). Then by (i), we have
        \begin{equation}\label{equ:quadratic_Gauss_sum_reduction_1}
          g(b,p^{r}) = pg(b,p^{r-2}) = \legendre{b}{p^{r-2}}pg(p^{r-2}) = \legendre{b}{p^{r-2}}g(p^{r}) = \legendre{b}{p^{r}}g(p^{r}).
        \end{equation}
        It now suffices to prove the claim when \(m = p^{r}q^{s}\) where \(q\) is another odd prime and \(s \ge 1\). Then by (ii) and \cref{equ:quadratic_Gauss_sum_reduction_1}, we compute
        \begin{align*}
          g(b,p^{r}q^{s}) &= g(bq^{s},p^{r})g(bp^{r},q^{s}) \\
          &= \legendre{bq^{s}}{p^{r}}\legendre{bp^{r}}{q^{s}}g(p^{r})g(q^{s}) \\
          &= \legendre{b}{p^{r}q^{s}}\legendre{q^{s}}{p^{r}}\legendre{p^{r}}{q^{s}}g(p^{r})g(q^{s}) \\
          &= \legendre{b}{p^{r}q^{s}}g(q^{s},p^{r})g(p^{r},q^{s}) \\
          &= \legendre{b}{p^{r}q^{s}}g(p^{r}q^{s}).
        \end{align*}
        This proves (iii).
      \end{enumerate}
    \end{proof}

    At last we can prove that our Ramanujan and quadratic Gauss sums agree for square-free \(m\):

    \begin{theorem}
      Suppose \(m \ge 1\) be square-free and odd and let \(\chi_{m}\) be the quadratic Dirichlet character given by the Jacobi symbol. Let \(b \in \mathbb{Z}\) such that \((b,m) = 1\). Then
      \[
        \tau(b,\chi_{m}) = g(b,m).
      \]
    \end{theorem}
    \begin{proof}
      The claim is obvious if \(m = 1\) because \(\tau(b,\chi_{1}) = 1\) and \(g(b,1) = 1\) so assume \(m > 1\). Since \(\chi_{m}\) is quadratic, it suffices to assume \(b = 1\) by \cref{prop:Gauss_sum_reduction} (ii) and \cref{prop:quadratic_Gauss_sum_reduction} (iii). Now let \(m = p_{1}p_{2} \cdots p_{k}\) be the prime decomposition of \(m\). Repeated application of \cref{prop:Gauss_sum_reduction} (iv) gives the first equality in the following chain:
      \begin{align*}
        \tau(\chi) &= \prod_{1 \le i < j \le k}\chi_{p_{i}}(p_{j})\chi_{p_{j}}(p_{i})\tau(\chi_{p_{i}})\tau(\chi_{p_{j}}) \\
        &= \prod_{1 \le i < j \le k}\chi_{p_{i}}(p_{j})\chi_{p_{j}}(p_{i})g(p_{i})g(p_{j}) \\
        &= \prod_{1 \le i < j \le k}g(p_{j},p_{i})g(p_{i},p_{j}) \\
        &= g(q).
      \end{align*}
      This completes the proof.
    \end{proof}

    Now let's turn to \cref{prop:quadratic_Gauss_sum_reduction} and the evaluation of the quadratic Gauss sum. \cref{prop:quadratic_Gauss_sum_reduction} (ii) and (iii) reduce the evaluation of \(g(b,m)\) for odd \(m\) and \((b,m) = 1\) to computing \(g(p)\) for \(p\) an odd prime. As with the Gauss sum, it is not difficult to compute the modulus of the quadratic Gauss sum:

    \begin{theorem}\label{thm:quadratic_Gauss_sum_modulus}
      Let \(m \ge 1\) be odd. Then
      \[
        |g(m)| = \sqrt{m}.
      \]
    \end{theorem}
    \begin{proof}
      By \cref{prop:quadratic_Gauss_sum_reduction} (ii), it suffices to assume \(m = p^{r}\) is a power of an odd prime. By Euclidean division write \(r = 2n+r'\) for some positive integer \(n\) and with \(r' = 0,1\) depending on if \(r\) is even or odd respectively. Then \cref{prop:quadratic_Gauss_sum_reduction} (i) implies
      \[
        |g(p^{r})|^{2} = p^{2n}|g(p^{r'})|^{2}.
      \]
      If \(r' = 0\) then \(2n = r\) so that \(p^{2n} = p^{r}\). Thus \(|g(p^{r})| = \sqrt{p^{r}}\). If \(r' = 1\) then \cref{thm:Gauss_sum_modulus,prop:Gauss_sum_equivalence_for_primes} together imply \(|g(p^{r'})|^{2} = p\) so that the right-hand side above is \(p^{2n+1} = p^{r}\) and again we have \(|g(p^{r})| = \sqrt{p^{r}}\).
    \end{proof}

    Accordingly, we define the \textit{epsilon factor} \(\e_{m}\) for any \(m \ge 1\) by
    \[
      \e_{m} = \frac{g(m)}{\sqrt{m}}.
    \]
    \cref{thm:quadratic_Gauss_sum_modulus} says that this value lies on the unit circle when \(m\) is odd. Thus the question of the evaluation of quadratic Gauss sums reduces to determining what the epsilon factor is. This was completely resolved and the original proof is due to Gauss in 1808 (see \cite{Gauss1808summatio}) while modern proofs use analytic techniques. We follow the analytic methods where the underlying idea is to rewrite \(g(m)\) in a form where \cref{thm:Poisson_summation_formula_Dirichlet-Jordan_test} can be applied. The precise statement and proof is given in the following:

    \begin{theorem}\label{thm:Gauss's_evaluation}
      Let \(m \ge 1\). Then
      \[
        \e_{m} = \begin{cases} (1+i) & \text{if \(m \equiv 0 \tmod{4}\)}, \\ 1 & \text{if \(m \equiv 1 \tmod{4}\)}, \\ 0 & \text{if \(m \equiv 2 \tmod{4}\)}, \\ i & \text{if \(m \equiv 3 \tmod{4}\)}. \end{cases}
      \]
    \end{theorem}
    \begin{proof}
      Consider the function
      \[
        f(x) = \begin{cases} e^{\frac{2\pi ix^{2}}{m}} & \text{if \(x \in [0,m]\),} \\ 0 & \text{if \(x \notin [0,m]\).} \end{cases}
      \]
      Observe that \(f(x)\) is absolutely integrable on \(\R\) as it is compactly supported. It is also continuously differentiable with a finite number of jump discontinuities (they are at \(x = 0\) and \(x = m\)). Then so is
      \[
        \asum_{n \in \Z}f(x+n)
      \]
      because finitely many of the summands are nonzero. In particular, this sum satisfies the Dirichlet-Jordan test. By \cref{thm:Poisson_summation_formula_Dirichlet-Jordan_test}, we have
      \[
        \asum_{n \in \Z}f(n) = \sum_{t \in \Z}(\mc{F}f)(t),
      \]
      where the \(\ast\) indicates that \(f(x+n)\) is meant to represent the average of the left-hand and right-hand limits at jump discontinuities. But from the definition of \(f(x)\), we find that
      \[
        \asum_{n \in \Z}f(n) = \sum_{a \tmod{m}}f(a) = g(m),
      \]
      and hence
      \[
        g(m) = \sum_{t \in \Z}(\mc{F}f)(t).
      \]
      It remains to compute the Fourier transform of \(f(x)\) which is
      \[
        (\mc{F}f)(t) = \int_{-\infty}^{\infty}f(x)e^{-2\pi itx} = \int_{0}^{m}e^{2\pi i\left(\frac{x^{2}}{m}-tx\right)}\,dx.
      \]
      By performing the change of variables \(x \mapsto \sqrt{m}x\), the last integral becomes
      \[
        \sqrt{m}\int_{0}^{\sqrt{m}}e^{2\pi i(x^{2}-t\sqrt{m}x)}\,dx.
      \]
      Complete the square in the exponent by observing
      \[
        x^{2}-t\sqrt{m}x = \left(x+\frac{t\sqrt{m}}{2}\right)^{2}-\frac{t^{2}m}{4},
      \]
      so that the previous integral is equal to
      \[
        \sqrt{m}e^{-\frac{2\pi it^{2}m}{4}}\int_{0}^{\sqrt{m}}e^{2\pi i\left(x+\frac{t\sqrt{m}}{2}\right)^{2}}\,dx.
      \]
      Changing variables \(x \mapsto x-\frac{t\sqrt{m}}{2}\) yields
      \[
        \sqrt{m}e^{-\frac{2\pi it^{2}m}{4}}\int_{\frac{t\sqrt{m}}{2}}^{\sqrt{m}+\frac{t\sqrt{m}}{2}}e^{2\pi ix^{2}}\,dx.
      \]
      As \(t \equiv 0,1 \tmod{4}\) according to if \(t\) is even or odd, we have
      \[
        \sum_{\substack{t \in \Z \\ \text{\(t\) even}}}(\mc{F}f)(t) = \sqrt{m}\int_{-\infty}^{\infty}e^{2\pi ix^{2}}\,dx \quad \text{and} \quad \sum_{\substack{t \in \Z \\ \text{\(t\) odd}}}(\mc{F}f)(t) = \sqrt{m}e^{-\frac{2\pi im}{4}}\int_{-\infty}^{\infty}e^{2\pi ix^{2}}\,dx.
      \]
      Whence
      \[
        g(m) = \sqrt{m}\left(1+e^{-\frac{2\pi im}{4}}\right)\int_{-\infty}^{\infty}e^{2\pi ix^{2}}\,dx.
      \]
      We now compute the remaining integral. As \(g(1) = 1\) and \(e^{-\frac{2\pi i}{4}} = -i\), taking \(m = 1\) yields
      \[
        \int_{-\infty}^{\infty}e^{2\pi ix^{2}}\,dx = \frac{1}{1-i}.
      \]
      Therefore
      \[
        \e_{m} = \frac{g(m)}{\sqrt{m}} = \frac{1+e^{-\frac{2\pi im}{4}}}{1-i} = \begin{cases} (1+i) & \text{if \(m \equiv 0 \tmod{4}\)}, \\ 1 & \text{if \(m \equiv 1 \tmod{4}\)}, \\ 0 & \text{if \(m \equiv 2 \tmod{4}\)}, \\ i & \text{if \(m \equiv 3 \tmod{4}\)}, \end{cases}
      \]
      as desired.
    \end{proof}

    As an immediate corollary, \cref{thm:Gauss's_evaluation} implies the evaluation of the epsilon factor \(\e_{\chi_{p}}\) where \(\chi_{p}\) is the quadratic Dirichlet character given by the Jacobi symbol for an odd prime \(p\):

    \begin{corollary}
      Let \(p\) be an odd prime and \(\chi_{p}\) be the quadratic Dirichlet character given by the Jacobi symbol. Then
      \[
        \e_{\chi_{p}} = \begin{cases} 1 & \text{if \(p \equiv 1 \tmod{4}\)}, \\ i & \text{if \(p \equiv 3 \tmod{4}\)}. \end{cases}
      \]
    \end{corollary}
    \begin{proof}
      The statement follows immediately from \cref{thm:Gauss's_evaluation,prop:Gauss_sum_equivalence_for_primes}.
    \end{proof}